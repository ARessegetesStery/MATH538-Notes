\documentclass{article}
\usepackage{../header}

\begin{document}

\Makepagesectionhead{MATH 538}{Lie Algebra}{ARessegetes Stery}

\tableofcontents  
\clearpage

\section{Lie Algebras}

\begin{definition}[Lie Algebra]
    Let $\F$ be a field. A \textbf{Lie Algebra} is a vector space $L$ over $\F$ equipped with a bilinear map $[\cdot, \cdot]: L \times L \to L$ (the \textbf{Lie bracket}) satisfying the following properties:
    \begin{itemize}
        \item \emph{Alternating Property}: $[x, x] = 0$ for all $x \in L$. (For $\fchar{\F} \neq 2$, this is equivalent to \underline{antisymmetry}: $[x, y] = -[y, x]$ for all $x, y \in L$.)
        \item \emph{Jacobi Identity}: $[x, [y, z]] + [y, [z, x]] + [z, [x, y]] = 0$ for all $x, y, z \in L$.
    \end{itemize}
    \ 
\end{definition}
\nogap
\begin{remark}
    Throughout this note, for simplicity we will assume that $\fchar{\F} = 0$. Nevertheless, the limitations on field characteristics will be pointed out when it is clear that the result will fall in certain cases.
\end{remark}

\begin{example}
    Consider $V = \F^n$. Notice that $\dim_\F (\End(V)) = \dim_\F (\Mat_n(\F)) = n^2$ as vector spaces over $\F$; and they are further isomorphic. We can further show that they are isomorphic as Lie algebras. 
\end{example}

\begin{proposition}
    Define the Lie bracket on $\End(V)$ by $[f, g] = fg - gf$ (with the product the composition of functions). Then $\End(V)$ is a Lie Algebra.
\end{proposition}

\begin{proof}
    It suffices to verify that Lie bracket satisfies the alternating property and the Jacobi identity. 
    \begin{itemize}
        \item \emph{Alternating Property}: $[f, f] = ff - ff = 0$.
        \item \emph{Jacobi Identity}: 
        \begin{align*}
            [f, [g, h]] + [g, [h, f]] + [h, [f, g]] &= [f, gh - hg] + [g, hf - fh] + [h, fg - gf] \\
            &= f(gh - hg) - (gh - hg)f + g(hf - fh) - (hf - fh)g + h(fg - gf) - (fg - gf)h \\
            &= fgh - fgh - ghf + hgf + ghf - hfg - hfg + fhg + fhg - fgh - gfh + gfh \\
            &= 0.
        \end{align*}
    \end{itemize}
    Bi-linearity results directly from the linearity of functions.
\end{proof}

\begin{notation}
    The Lie algebra $(\End(V), [\cdot, \cdot])$ is denoted by $\liealg{gl}(V)$.
\end{notation}
\nogap
\begin{example}
    Let $V = \R^n$. Then as vector spaces $\End(V) \simeq \Mat_n(\R) \simeq \liealg{gl}(V)$ where $[A, B] = AB - BA$ for $A, B \in \Mat_n(\R)$.
\end{example}

The Lie Algebra $\End(V) \simeq \Mat_n(V) \simeq \liealg{gl}(V)$ has a basis $\{ e_{ij} \}_{i, j = 1}^n$, where $\{e_{ij}\}$ represents the matrix with all zero entries except for $1$ in the $(i, j)$-th entry.

\begin{definition}[Lie Subalgebra]
    A \textbf{Lie Subalgebra} $K \subseteq L$ is a subspace of s.t. for all $x, y \in K$, $[x, y] \in K$.
\end{definition}
\nogap
\begin{definition}[Linear Lie Algebra]
    Any subalgebra of $\liealg{gl}(V)$ is called a \textbf{linear Lie algebra}.
\end{definition}

\begin{theorem}[Ado-Iwasawa]\label{thm:ado-iwasawa}
    Every finite dimensional Lie algebra is isomorphic to a linear Lie algebra.
\end{theorem}

The proof of this statement requires more structure and will be deferred; for now we will use this results without proof as this theorem is really powerful.

\TODO{search for proof}

\begin{example}[Classical Lie Algebra]\label{ex: classical Lie algebra}
    The following examples of Lie algebra constitute the \textbf{classical Lie algebra} which are the bulk of existing Lie algebras. As expected, they are closely related to matrices.
    \begin{enumerate}
        \item Let $V = \F^{n+1}$. The \textbf{special linear algebra} $A_n$ is
        \[
            A_n = \liealg{sl}_{n+1}(V) = \{ g \in \liealg{gl}_{n+1}(V) \mid \Tr{g} = 0 \}
        \]
        As a vector space over $\F$ it has dimension $(n^2 + 2n)$.
        \item Let $V = \F^{2n}$. The \textbf{symplectic algebra} $C_n$ is
        \[
            C_n = \liealg{sp}_{2n}(V) = \{ g \in \liealg{gl}_{2n}(V) \mid sg + g^T s = 0 \}, \quad \text{where } s = \begin{pmatrix} 0 & \Id_n \\ -\Id_n & 0 \end{pmatrix}
        \]
        Considering its dimension, writing also $g \in \liealg{gl}_{2n}(V)$ as $n$-by-$n$ blocks, we have
        \[
            g = 
            \left(\begin{array}{c|c}
                g_1 & g_2 \\
                \hline
                g_3 & g_4
            \end{array}\right)
            \quad
            \implies
            \quad
            sg + g^T s = 
            \left(\begin{array}{c|c}
                g_3 - g_3^T & g_4 + g_1^T \\
                \hline
                g_1 + g_4^T & g_2 - g_2^T
            \end{array}\right)
            = 0
        \]
        Further notice that $(g_4 + g_1^T) = (g_1 + g_4^T)^T$. Then the condition of matrix vanishing becomes equivalent to $g_3 = g_3^T$, $g_2 = -g_2^T$ and $g_1 = -g_4^T$. For a symmetric matrix its dimension is $\frac{1}{2}n(n+1)$, and arbitrary $g_1$ fixes $g_4$, giving dimension $n$. Summing together gives the $\dim_\F C_n = 2n^2 + n$.
        \item Let $V = \F^{2n+1}$. The \textbf{(odd) orthogonal algebra} $B_n$ is
        \[
            B_n = \liealg{o}_{2n+1}(V) = \{ g \in \liealg{gl}_{2n+1}(V) \mid sg + g^T s = 0 \}, \quad \text{where } s = \begin{pmatrix} 1 & 0 & 0 \\ 0 & 0 & \Id_n \\ 0 & \Id_n & 0 \end{pmatrix}
        \]
        Considering its dimension, write also $g \in \liealg{gl}_{2n+1}(V)$ in block form:
        \[
            g = 
            \begin{pmatrix}
                x              & a_{1 \times n} & b_{1 \times n} \\
                c_{n \times 1} & m_{n \times n} & n_{n \times n} \\
                d_{n \times 1} & p_{n \times n} & q_{n \times n}
            \end{pmatrix}
            \quad
            \implies
            \quad
            sg + g^T s =
            \begin{pmatrix}
                2x      & a + d^T & b + c^T \\
                a^T + d & p + p^T & q + m^T \\
                b^T + c & m + q^T & n + n^T
            \end{pmatrix}
            = 0
        \]
        This translates to equalities
        \[
            2x = 0, \quad
            a + d^T = (a^T + d)^T = 0, \quad
            b + c^T = (b^T + c)^T = 0, \quad
            q + m^T = (m + q^T)^T = 0, \quad 
            p + p^T = 0, \quad
            n + n^T = 0
        \]
        This gives $x = 0$. Fixing $a$ and $b$ fixes $d$ and $c$, of which there are $2n$ choices. Fixing $m$ fixes $q$, of which there are $n^2$ choices. It is also required that both $n$ and $p$ are anti-symmetric matrices, i.e. they have zeros on the diagonal, and fixing upper triangle elements fixes the whole matrix, of which there are $\frac{1}{2}n(n-1)$ choices. Then
        \[
            \dim_\F B_n = 2n + n^2 + 2 \cdot \frac{1}{2} n(n-1) = 2n^2 + n
        \]
        \item Let $V = \F^{2n}$. The \textbf{(even) orthogonal algebra} $D_n$ is
        \[
            D_n = \liealg{o}_{2n}(V) = \{ g \in \liealg{gl}_{2n}(V) \mid sg + g^T s = 0 \}, \quad \text{where } s = \begin{pmatrix} 0 & \Id_n \\ \Id_n & 0 \end{pmatrix}
        \]
        Considering its dimension, use the similar strategy as above. Write $g \in \liealg{gl}_{2n}(V)$ in block form:
        \[
            g = 
            \begin{pmatrix}
                m & n \\
                p & q
            \end{pmatrix}
            \quad \implies \quad
            sg + g^T s =
            \begin{pmatrix}
                p + p^T & q + m^T \\
                m + q^T & n + n^T
            \end{pmatrix}
            = 0
        \]
        The equality translates to the following conditions
        \[
            p + p^T = 0, \quad
            q + m^T = (m + q^T)^T = 0, \quad
            n + n^T = 0
        \]
        Fixing $p$ fixes $m$, and both $p$ and $n$ are anti-symmetric matrices, which have dimension $\frac{1}{2}n(n-1)$ over $\F$. This gives
        \[
            \dim_\F D_n = n^2 + 2 \cdot \frac{1}{2}n(n-1) = 2n^2 - n
        \] 
    \end{enumerate}
    \ 
\end{example}

\section{Structure of Lie Algebras}

\textstart
Now we turn to discuss the algebraic structures of Lie algebra. Being an algebra itself, we are interested in some special properties, analogous to the study of groups or rings. In general, some of the definitions and theorems can be rephrased in a categorical sense, and thereby applies to all such similar structures.

\begin{definition}[$\F$-Algebra]
    Let $\F$ be a field. An $\F$-algebra is a vector space $U$ over $\F$ equipped with a bilinear map $(\cdot): U \times U \to U$, denoted $(u_1, u_2) \mapsto u_1 u_2$.
\end{definition}
\nogap
\begin{remark}
    In this sense, the Lie algebra $L$ (by definition is an $\F$-vector space) is indeed an $\F$-algebra (compatible with its name), with the bilinear map $(\cdot)$ being the Lie bracket.
\end{remark}

\begin{definition}[Derivation]
    Let $U$ be a vector space over field $\F$. A \textbf{derivation} of $U$ is $S \in \liealg{gl}(U)$ s.t. for all $a, b \in U$, $S(ab) = a S(b) + S(a) b$ (i.e. satisfies the \underline{Leibniz Rule}). The set of derivations on $U$ is denoted by $\Der(U)$.
\end{definition}

\begin{remark}
    $\Der(U)$ is a subalgebra of $\liealg{gl}(U)$. By definition, it suffices to verify that for all $S, T \in \Der(U)$, $[S, T] \in \Der(U)$. Check that the Leibniz rule is satisfied:
    \begin{align*}
        [S, T](ab) &= S(T(ab)) - T(S(ab)) \\
        & = S(aT(b) + T(a)b) - T(aS(b) + S(a)b) \\
        & = aS(T(b)) + bS(T(a)) - aT(S(b)) - bT(S(a)) \\
        & = a[S, T](b) + b[S, T](a)
    \end{align*}
\end{remark}

\begin{definition}[Adjoint Representation]
    The map $\ad: L \to \Der(L)$ is the \textbf{adjoint representation}, defined by $\ad(x)(y) = [x, y]$
\end{definition}

\textstart
An immediate thought is to verify that indeed $\ad(x)$ gives a derivation. Recall that the product defined in Lie algebra is the Lie bracket. Therefore, to verify the derivation property it suffices to check whether we have the equality
\[
    \ad(x)([y, z]) = [\ad(x)(y), z] + [y, \ad(x)(z)]
\]
Applying the definition and manipulating the terms, we get the Jacobi Identity, which verifies the equality:
\begin{align*}
    \ad(x)([y, z])
    & = [x, [y, z]] = -[y, [z, x]] - [z, [x, y]] = -[y, \ad(x)(z)] - [z, \ad(x)(y)] \\
    & = [\ad(x)(y), z] + [y, \ad(x)(z)]
\end{align*}

\begin{definition}[Structure Constants]
    Let $\{x_1, \dots, x_n\}$ be a basis of $L$ as an $\F$-vector space. The \textbf{structure constants} of $L$ are the coefficients $c_{ij}^k$ s.t. $[x_i, x_j] = \sum_{k = 1}^n c_{ij}^k x_k$.
\end{definition}

\textstart
It is clear that the structure constants are specified solely by the definition of Lie bracket, which is the sole extra structure given to any Lie algebra apart from the underlying vector space structure.

\begin{remark}
    Using the structure constants we can rewrite the Jacobi Identity. Given a basis of $L$ over $\F$ and its corresponding structure constants $a_{ij}^k \in \F$, the Jacobi Identity can be written as
    \[
        \sum_{k = 1}^n \left( a_{ij}^k a_{k \ell}^m + a_{j\ell}^k a_{ki}^m + a_{\ell i}^k a_{kj}^m \right) = 0, \quad \forall i, j, \ell, m \in \{1, \dots, n\}
    \]
\end{remark}

\begin{definition}[Abelian]
    A Lie algebra $L$ is \textbf{abelian} if $[x, y] = 0$ for all $x, y \in L$.
\end{definition}

\begin{definition}[Ideal]
    Given a Lie algebra $L$, a subspace $I \subseteq L$ is an \textbf{ideal} if for all $x \in I$ and $y \in L$, $[x, y] \in I$.
\end{definition}
\nogap
\begin{remark}
    This kind of ``absorbing'' property is analogous to the normal subgroup in group theory. Considering the Lie bracket as a multiplication on a ring, this is compatible with the definition of an ideal in a ring.
\end{remark}
\nogap
\begin{remark}
    Given a Lie algebra $L$, if both $I$ and $J$ are ideals in $L$, then $I + J$, $I \cap J$ and $[I, J]$ are also ideals.
\end{remark}

\textstart
With the similar formulation of structure preserving sub-objects, we have similar structures as in group or ring theory.

\begin{definition}[Quotient]
    Given a Lie algebra $L$ and an ideal $I \subseteq L$, the \textbf{quotient} $L/I$ is the vector space $L/I = \{ x + I \mid x \in L \}$ with the Lie bracket $[x + I, y + I] = [x, y] + I$.
\end{definition}
\nogap
\begin{definition}[Center]
    Given a Lie algebra $L$, the \textbf{center} of $L$ is defined as $Z(L) = \{ z \in L \mid [z, x] = 0 \text{ for all } x \in L \}$.
\end{definition}
\nogap
\begin{definition}[Derived Algebra]
    Given a Lie algebra $L$, the \textbf{derived algebra} of $L$ is $[L, L] = L' = L^{(1)}$, where $[L, L]$ can be interpreted as the set $[L, L] := \{ [x, y] \mid x, y \in L \}$
    By definition this is an ideal of $L$.
\end{definition}
\nogap
\begin{definition}[Simple]
    A Lie algebra $L$ is \textbf{simple} if $[L, L] \neq 0$ (i.e. it is non-trivial), and the only ideals of $L$ are trivial ($0$ and $L$).
\end{definition}

\begin{example}
    Let $L = \liealg{gl}_n(\F)$. Then $[e_{ij}, e_{k\ell}] = \delta_{jk} e_{i \ell} - \delta_{\ell i} e_{kj}$. Setting $k = j$ and $i = \ell$ gives $\Tr([e_{ij}, e_{k\ell}]) = 0$. By definitions in Example \ref{ex: classical Lie algebra}, $L' = [L, L] \simeq \liealg{sl}_n(F)$. Since specifically $\liealg{sl}_n(\F) \subseteq \liealg{gl}_n(\F)$ which gives an ideal, $\liealg{gl}_n(\F)$ is not simple. In fact $\liealg{sl}_n(\F)$ is simple, but proving this result requires more constructions.
\end{example}

\begin{definition}[Normalizer]
    Let $L$ be a Lie algebra and $K$ a subalgebra of $L$. The \textbf{normalizer} of $K$ is defined by $N_L(K) := \{ x \in L \mid [x, K] \subseteq K \}$, the largest subalgebra of $L$ in which $K$ is an ideal.
\end{definition}
\nogap
\begin{definition}[Centralizer]
    Let $L$ be a Lie algebra and $X$ an arbitrary set. The \textbf{centralizer} of $X$ is defined by $C_L(X) := \{ x \in L \mid [x, X] = 0 \}$.
\end{definition}

\end{document}
