\documentclass{article}
\usepackage{../header}

\begin{document}

\Makepagesectionhead{MATH 538}{Lie Algebra}{ARessegetes Stery}

\tableofcontents  
\clearpage

\section{Lie Algebra}

\begin{definition}[Lie Algebra]
    Let $\F$ be a field. A \textbf{Lie Algebra} is a vector space $L$ over $\F$ equipped with a bilinear map $[\cdot, \cdot]: L \times L \to L$ (the \textbf{Lie Bracket}) satisfying the following properties:
    \begin{itemize}
        \item \emph{Alternating Property}: $[x, x] = 0$ for all $x \in L$. (For $\fchar{\F} \neq 2$, this is equivalent to \underline{antisymmetry}: $[x, y] = -[y, x]$ for all $x, y \in L$.)
        \item \emph{Jacobi Identity}: $[x, [y, z]] + [y, [z, x]] + [z, [x, y]] = 0$ for all $x, y, z \in L$.
    \end{itemize}
\end{definition}

\begin{example}
    Consider $V = \F^n$. Notice that $\dim (\End(V)) = \dim(\Mat_n(\F)) = n^2$ as vector spaces over $\F$; and they are further isomorphic. We can further show that they are isomorphic as Lie algebras. 
\end{example}

\begin{proposition}
    Define the Lie bracket on $\End(V)$ by $[f, g] = fg - gf$ (with the product the composition of functions). Then $\End(V)$ is a Lie Algebra.
\end{proposition}

\begin{proof}
    It suffices to verify that Lie bracket satisfies the alternating property and the Jacobi identity. 
    \begin{itemize}
        \item \emph{Alternating Property}: $[f, f] = ff - ff = 0$.
        \item \emph{Jacobi Identity}: 
        \begin{align*}
            [f, [g, h]] + [g, [h, f]] + [h, [f, g]] &= [f, gh - hg] + [g, hf - fh] + [h, fg - gf] \\
            &= f(gh - hg) - (gh - hg)f + g(hf - fh) - (hf - fh)g + h(fg - gf) - (fg - gf)h \\
            &= fgh - fgh - ghf + hgf + ghf - hfg - hfg + fhg + fhg - fgh - gfh + gfh \\
            &= 0.
        \end{align*}
    \end{itemize}
    Bi-linearity results directly from the linearity of functions.
\end{proof}

\begin{notation}
    The Lie algebra $(\End(V), [\cdot, \cdot])$ is denoted by $\liealg{gl}(V)$.
\end{notation}
\nogap
\begin{example}
    Let $V = \R^n$. Then as vector spaces $\End(V) \simeq \Mat_n(\R) \simeq \liealg{gl}(V)$ where $[A, B] = AB - BA$ for $A, B \in \Mat_n(\R)$.
\end{example}

\begin{definition}[Lie Subalgebra]
    A \textbf{Lie Subalgebra} $K \subseteq L$ is a subspace of s.t. for all $x, y \in K$, $[x, y] \in K$.
\end{definition}
\nogap
\begin{definition}[Linear Algebra]
    Any subalgebra of $\liealg{gl}(V)$ is called a \textbf{linear algebra}.
\end{definition}

\section{An Algebraic Perspective on Lie Algebra}

\end{document}
