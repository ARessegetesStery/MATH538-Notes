\documentclass{article}
\usepackage{../header}

\begin{document}

\Makepagesectionhead{MATH 538}{Lie Algebra}{ARessegetes Stery}

\tableofcontents  
\clearpage

\section{Overview of Lie Algebras}

\begin{definition}[Lie Algebra]
    Let $\F$ be a field. A \textbf{Lie Algebra} is a vector space $L$ over $\F$ equipped with a bilinear map $[\cdot, \cdot]: L \times L \to L$ (the \textbf{Lie bracket}) satisfying the following properties:
    \begin{itemize}
        \item \emph{Alternating Property}: $[x, x] = 0$ for all $x \in L$. (For $\fchar{\F} \neq 2$, this is equivalent to \underline{antisymmetry}: $[x, y] = -[y, x]$ for all $x, y \in L$.)
        \item \emph{Jacobi Identity}: $[x, [y, z]] + [y, [z, x]] + [z, [x, y]] = 0$ for all $x, y, z \in L$.
    \end{itemize}
    The \textbf{dimension} of a Lie algebra is the dimension of the underlying vector space.
\end{definition}
\nogap
\begin{remark}
    Throughout this note, for simplicity we will assume that $\fchar{\F} = 0$. Nevertheless, the limitations on field characteristics will be pointed out when it is clear that the result will fall for certain cases.
\end{remark}

\begin{example}
    Consider $V = \F^n$. Notice that $\dim_\F (\End(V)) = \dim_\F (\Mat_n(\F)) = n^2$ as vector spaces over $\F$; and they are further isomorphic. We can further show that they are isomorphic as Lie algebras. 
\end{example}

\begin{proposition}
    Define the Lie bracket on $\End(V)$ by $[f, g] = fg - gf$ (with the product the composition of functions). Then $\End(V)$ is a Lie Algebra.
\end{proposition}

\begin{proof}
    It suffices to verify that Lie bracket satisfies the alternating property and the Jacobi identity. 
    \begin{itemize}
        \item \emph{Alternating Property}: $[f, f] = ff - ff = 0$.
        \item \emph{Jacobi Identity}: 
        \begin{align*}
            [f, [g, h]] + [g, [h, f]] + [h, [f, g]] &= [f, gh - hg] + [g, hf - fh] + [h, fg - gf] \\
            &= f(gh - hg) - (gh - hg)f + g(hf - fh) - (hf - fh)g + h(fg - gf) - (fg - gf)h \\
            &= fgh - fgh - ghf + hgf + ghf - hfg - hfg + fhg + fhg - fgh - gfh + gfh \\
            &= 0.
        \end{align*}
    \end{itemize}
    Bi-linearity results directly from the linearity of functions.
\end{proof}

\begin{notation}
    The Lie algebra $(\End(V), [\cdot, \cdot])$ is denoted by $\gl(V)$.
\end{notation}
\nogap
\begin{example}
    Let $V = \R^n$. Then as vector spaces $\End(V) \simeq \Mat_n(\R) \simeq \gl(V)$ where $[A, B] = AB - BA$ for $A, B \in \Mat_n(\R)$.
\end{example}

The Lie Algebra $\End(V) \simeq \Mat_n(V) \simeq \gl(V)$ has a basis $\{ e_{ij} \}_{i, j = 1}^n$, where $\{e_{ij}\}$ represents the matrix with all zero entries except for $1$ in the $(i, j)$-th entry.

\begin{definition}[Lie Subalgebra]
    A \textbf{Lie Subalgebra} $K \subseteq L$ is a subspace of s.t. for all $x, y \in K$, $[x, y] \in K$.
\end{definition}
\nogap
\begin{definition}[Linear Lie Algebra]\label{def: linear lie algebra}
    Any subalgebra of $\gl(V)$ is called a \textbf{linear Lie algebra}.
\end{definition}

\begin{theorem}[Ado-Iwasawa]\label{thm:ado-iwasawa}
    Every finite dimensional Lie algebra is isomorphic to a linear Lie algebra.
\end{theorem}

\textstart
The proof of this statement requires more structure and will be deferred.

\TODO{search for proof}

\begin{example}[Classical Lie Algebra]\label{ex: classical Lie algebra}
    The following examples of Lie algebra constitute the \textbf{classical Lie algebra} which are the bulk of existing Lie algebras. As expected, they are closely related to matrices.
    \begin{enumerate}
        \item Let $V = \F^{n+1}$. The \textbf{special linear algebra} $A_n$ is
        \[
            A_n = \liealg{sl}_{n+1}(V) = \{ g \in \gl_{n+1}(V) \mid \Tr{g} = 0 \}
        \]
        As a vector space over $\F$ it has dimension $(n^2 + 2n)$.
        \item Let $V = \F^{2n}$. The \textbf{symplectic algebra} $C_n$ is
        \[
            C_n = \liealg{sp}_{2n}(V) = \{ g \in \gl_{2n}(V) \mid sg + g^T s = 0 \}, \quad \text{where } s = \begin{pmatrix} 0 & \Id_n \\ -\Id_n & 0 \end{pmatrix}
        \]
        Considering its dimension, writing also $g \in \gl_{2n}(V)$ as $n$-by-$n$ blocks, we have
        \[
            g = 
            \left(\begin{array}{c|c}
                g_1 & g_2 \\
                \hline
                g_3 & g_4
            \end{array}\right)
            \quad
            \implies
            \quad
            sg + g^T s = 
            \left(\begin{array}{c|c}
                g_3 - g_3^T & g_4 + g_1^T \\
                \hline
                g_1 + g_4^T & g_2 - g_2^T
            \end{array}\right)
            = 0
        \]
        Further notice that $(g_4 + g_1^T) = (g_1 + g_4^T)^T$. Then the condition of matrix vanishing becomes equivalent to $g_3 = g_3^T$, $g_2 = -g_2^T$ and $g_1 = -g_4^T$. For a symmetric matrix its dimension is $\frac{1}{2}n(n+1)$, and arbitrary $g_1$ fixes $g_4$, giving dimension $n$. Summing together gives the $\dim_\F C_n = 2n^2 + n$.
        \item Let $V = \F^{2n+1}$. The \textbf{(odd) orthogonal algebra} $B_n$ is
        \[
            B_n = \liealg{o}_{2n+1}(V) = \{ g \in \gl_{2n+1}(V) \mid sg + g^T s = 0 \}, \quad \text{where } s = \begin{pmatrix} 1 & 0 & 0 \\ 0 & 0 & \Id_n \\ 0 & \Id_n & 0 \end{pmatrix}
        \]
        Considering its dimension, write also $g \in \gl_{2n+1}(V)$ in block form:
        \[
            g = 
            \begin{pmatrix}
                x              & a_{1 \times n} & b_{1 \times n} \\
                c_{n \times 1} & m_{n \times n} & n_{n \times n} \\
                d_{n \times 1} & p_{n \times n} & q_{n \times n}
            \end{pmatrix}
            \quad
            \implies
            \quad
            sg + g^T s =
            \begin{pmatrix}
                2x      & a + d^T & b + c^T \\
                a^T + d & p + p^T & q + m^T \\
                b^T + c & m + q^T & n + n^T
            \end{pmatrix}
            = 0
        \]
        This translates to equalities
        \[
            2x = 0, \quad
            a + d^T = (a^T + d)^T = 0, \quad
            b + c^T = (b^T + c)^T = 0, \quad
            q + m^T = (m + q^T)^T = 0, \quad 
            p + p^T = 0, \quad
            n + n^T = 0
        \]
        This gives $x = 0$. Fixing $a$ and $b$ fixes $d$ and $c$, of which there are $2n$ choices. Fixing $m$ fixes $q$, of which there are $n^2$ choices. It is also required that both $n$ and $p$ are anti-symmetric matrices, i.e. they have zeros on the diagonal, and fixing upper triangle elements fixes the whole matrix, of which there are $\frac{1}{2}n(n-1)$ choices. Then
        \[
            \dim_\F B_n = 2n + n^2 + 2 \cdot \frac{1}{2} n(n-1) = 2n^2 + n
        \]
        \item Let $V = \F^{2n}$. The \textbf{(even) orthogonal algebra} $D_n$ is
        \[
            D_n = \liealg{o}_{2n}(V) = \{ g \in \gl_{2n}(V) \mid sg + g^T s = 0 \}, \quad \text{where } s = \begin{pmatrix} 0 & \Id_n \\ \Id_n & 0 \end{pmatrix}
        \]
        Considering its dimension, use the similar strategy as above. Write $g \in \gl_{2n}(V)$ in block form:
        \[
            g = 
            \begin{pmatrix}
                m & n \\
                p & q
            \end{pmatrix}
            \quad \implies \quad
            sg + g^T s =
            \begin{pmatrix}
                p + p^T & q + m^T \\
                m + q^T & n + n^T
            \end{pmatrix}
            = 0
        \]
        The equality translates to the following conditions
        \[
            p + p^T = 0, \quad
            q + m^T = (m + q^T)^T = 0, \quad
            n + n^T = 0
        \]
        Fixing $p$ fixes $m$, and both $p$ and $n$ are anti-symmetric matrices, which have dimension $\frac{1}{2}n(n-1)$ over $\F$. This gives
        \[
            \dim_\F D_n = n^2 + 2 \cdot \frac{1}{2}n(n-1) = 2n^2 - n
        \] 
    \end{enumerate}
    \ 
\end{example}

\section{Analogy with other Algebraic Structures}

\textstart
Now we turn to discuss the algebraic structures of Lie algebra. Being an algebra itself, we are interested in some special properties, analogous to the study of groups or rings. In general, some of the definitions and theorems can be rephrased in a categorical sense, and thereby applies to all such similar structures.

\begin{definition}[$\F$-Algebra]
    Let $\F$ be a field. An $\F$-algebra is a vector space $U$ over $\F$ equipped with a bilinear map $(\cdot): U \times U \to U$, denoted $(u_1, u_2) \mapsto u_1 u_2$.
\end{definition}
\nogap
\begin{remark}
    In this sense, the Lie algebra $L$ (by definition is an $\F$-vector space) is indeed an $\F$-algebra (compatible with its name), with the bilinear map $(\cdot)$ being the Lie bracket.
\end{remark}

\begin{definition}[Derivation]
    Let $U$ be a vector space over field $\F$. A \textbf{derivation} of $U$ is $S \in \gl(U)$ s.t. for all $a, b \in U$, $S(ab) = a S(b) + S(a) b$ (i.e. satisfies the \underline{Leibniz Rule}). The set of derivations on $U$ is denoted by $\Der(U)$.
\end{definition}

\begin{remark}
    $\Der(U)$ is a subalgebra of $\gl(U)$. By definition, it suffices to verify that for all $S, T \in \Der(U)$, $[S, T] \in \Der(U)$. Check that the Leibniz rule is satisfied:
    \begin{align*}
        [S, T](ab) &= S(T(ab)) - T(S(ab)) \\
        & = S(aT(b) + T(a)b) - T(aS(b) + S(a)b) \\
        & = aS(T(b)) + bS(T(a)) - aT(S(b)) - bT(S(a)) \\
        & = a[S, T](b) + b[S, T](a)
    \end{align*}
\end{remark}

\begin{definition}[Adjoint Representation]
    The map $\ad: L \to \Der(L)$ is the \textbf{adjoint representation}, defined by $\ad(x)(y) = [x, y]$
\end{definition}

\textstart
It turns out that $\ad(x)$ gives a derivation. Recall that the product defined in Lie algebra is the Lie bracket. Therefore, to verify the derivation property it suffices to check whether we have the equality
\[
    \ad(x)([y, z]) = [\ad(x)(y), z] + [y, \ad(x)(z)]
\]
Applying the definition and manipulating the terms, we get the Jacobi Identity, which verifies the equality:
\begin{align*}
    \ad(x)([y, z])
    & = [x, [y, z]] = -[y, [z, x]] - [z, [x, y]] = -[y, \ad(x)(z)] - [z, \ad(x)(y)] \\
    & = [\ad(x)(y), z] + [y, \ad(x)(z)]
\end{align*}

\begin{definition}[Structural Constants]
    Let $\{x_1, \dots, x_n\}$ be a basis of $L$ as an $\F$-vector space. The \textbf{structural constants} of $L$ are the coefficients $c_{ij}^k$ s.t. $[x_i, x_j] = \sum_{k = 1}^n c_{ij}^k x_k$.
\end{definition}

\textstart
It is clear that the structural constants are specified solely by the definition of Lie bracket, which is the sole extra structure given to any Lie algebra apart from the underlying vector space structure.

\begin{remark}
    Using the structural constants we can rewrite the Jacobi Identity. Given a basis of $L$ over $\F$ and its corresponding structural constants $a_{ij}^k \in \F$, the Jacobi Identity can be written as
    \[
        \sum_{k = 1}^n \left( a_{ij}^k a_{k \ell}^m + a_{j\ell}^k a_{ki}^m + a_{\ell i}^k a_{kj}^m \right) = 0, \quad \forall i, j, \ell, m \in \{1, \dots, n\}
    \]
\end{remark}

\begin{definition}[Abelian]
    A Lie algebra $L$ is \textbf{abelian} if $[x, y] = 0$ for all $x, y \in L$.
\end{definition}

\begin{definition}[Ideal]
    Given a Lie algebra $L$, a subspace $I \subseteq L$ is an \textbf{ideal} if for all $x \in I$ and $y \in L$, $[x, y] \in I$.
\end{definition}
\nogap
\begin{remark}
    This kind of ``absorbing'' property is analogous to the normal subgroup in group theory. Considering the Lie bracket as a multiplication on a ring, this is also compatible with the definition of an ideal in a ring.
\end{remark}
\nogap
\begin{remark}\label{rmk: extending ideals}
    Given a Lie algebra $L$, if both $I$ and $J$ are ideals in $L$, then $I + J$, $I \cap J$ and $[I, J]$ are also ideals.
\end{remark}

\textstart
With the similar formulation of structure preserving sub-objects, we have similar structures as in group or ring theory.

\begin{definition}[Quotient]
    Given a Lie algebra $L$ and an ideal $I \subseteq L$, the \textbf{quotient} $L/I$ is the vector space $L/I = \{ x + I \mid x \in L \}$ with the Lie bracket $[x + I, y + I] = [x, y] + I$.
\end{definition}
\nogap
\begin{definition}[Center]
    Given a Lie algebra $L$, the \textbf{center} of $L$ is defined as $Z(L) = \{ z \in L \mid [z, x] = 0 \text{ for all } x \in L \}$.
\end{definition}
\nogap
\begin{definition}[Derived Algebra]
    Given a Lie algebra $L$, the \textbf{derived algebra} of $L$ is $[L, L] = L' = L^{(1)}$, where $[L, L]$ can be interpreted as the set $[L, L] := \{ [x, y] \mid x, y \in L \}$
    By definition this is an ideal of $L$.
\end{definition}
\nogap
\begin{definition}[Simple]
    A Lie algebra $L$ is \textbf{simple} if $[L, L] \neq 0$ (i.e. it is non-trivial), and the only ideals of $L$ are trivial ($0$ and $L$).
\end{definition}

\begin{example}\label{ex: simple lie algebra}
    Let $L = \gl_n(\F)$. Then $[e_{ij}, e_{k\ell}] = \delta_{jk} e_{i \ell} - \delta_{\ell i} e_{kj}$. Setting $k = j$ and $i = \ell$ gives $\Tr([e_{ij}, e_{k\ell}]) = 0$. By definitions in Example \ref{ex: classical Lie algebra}, $L^{(1)} = [L, L] \simeq \liealg{sl}_n(\F)$. Since specifically $\liealg{sl}_n(\F) \subseteq \gl_n(\F)$ which gives an ideal, $\gl_n(\F)$ is not simple. In fact $\liealg{sl}_n(\F)$ is simple, but proving this result requires more constructions.
\end{example}

\begin{definition}[Normalizer]
    Let $L$ be a Lie algebra and $K$ a subalgebra of $L$. The \textbf{normalizer} of $K$ is defined by $N_L(K) := \{ x \in L \mid [x, K] \subseteq K \}$, the largest subalgebra of $L$ in which $K$ is an ideal.
\end{definition}
\nogap
\begin{definition}[Centralizer]
    Let $L$ be a Lie algebra and $X$ an arbitrary set. The \textbf{centralizer} of $X$ is defined by $C_L(X) := \{ x \in L \mid [x, X] = 0 \}$, the largest subalgebra of $L$ in which $K$ is in the center.
\end{definition}

\begin{definition}[Morphism]
    Let $L$ and $L'$ be two Lie algebras over $\F$. A \textbf{homomorphism} is an $\F$-linear transformation $\phi: L \to L'$ s.t. $\phi([x, y]) = [\phi(x), \phi(y)]$, i.e. it commutes with the multiplication, which is the Lie bracket here.

    Adopting general categorical nomenclature, a homomorphism is a \textbf{monomorphism} if it is injective, an \textbf{epimorphism} if it is surjective, an \textbf{isomorphism} if it is bijective, and an \textbf{automorphism} if it is an isomorphism from $L$ to itself.
\end{definition}
\nogap
\begin{remark}\label{rmk: extending morphism results}
    With the definition of morphisms, we can easily translate common results from group or ring theory to Lie algebras. For all homomorphism $\phi: L \to L'$:
    \begin{enumerate}
        \item Both $\ker \phi$ and $\im \phi$ are ideals of $L$.
        \item \emph{(First Isomorphism Theorem)} $\im \phi \simeq L/\ker \phi$.
        \item \emph{(Second Isomorphism Theorem)} Given two ideals $I, J \subseteq L$, by Remark \ref{rmk: extending ideals} $I + J$ is also an ideal of $L$. We have the isomorphism $(I + J)/J \simeq I/I \cap J$.
        \item \emph{(Third Isomorphism Theorem)} Let $I \subseteq J \subseteq L$ where both $I$ and $J$ are ideals of $L$. Then $L/J \simeq (L/I)/(J/I)$.
        \item \emph{(Fourth Isomorphism Theorem, Correspondence)} Let $I \subseteq \ker \phi$ be an ideal in $\ker \phi$. Then there exists a unique homomorphism $\psi: L/I \to L'$, i.e. making the following diagram commute:
        
        \begin{minipage}{\linewidth}
            \centering
            \begin{tikzcd}
                L \arrow[rr, "\phi"] \arrow[rrdd] & & L' \\
                & & \\
                & & L/I \arrow[uu, dashed, "\psi"]
            \end{tikzcd}
        \end{minipage}
    \end{enumerate}
    \ 
\end{remark}

\begin{definition}[Representation]
    Let $L$ and $L'$ be Lie algebras over $\F$, and $V$ a vector space over $\F$. A \textbf{representation} of $L$ on $V$ is a homomorphism $\rho: L \to \gl(V)$.
\end{definition}

\begin{example}[Adjoint Representation]
    For a Lie algebra $L$, recall that the adjoint map is $\ad: L \to \gl(L)$, $\ad(x)(y) = [x, y]$. Since $L$ is itself a vector space over $\F$, letting $V = L$ we have a representation. This is indeed a homomorphism as
    \begin{align*}
        [\ad(x), \ad(y)](z) 
        & = \ad(x) \ad(y)(z) - \ad(y) \ad(x)(z) = [x, [y, z]] - [y, [x, z]] \\
        & = [x, [y, z]] + [y, [z, x]] = -[z, [x, y]] = [[x, y], z] \\
        & = \ad([x, y])(z)
    \end{align*}
\end{example}

\begin{proposition}
    Any simple Lie algebra is isomorphic to a linear Lie algebra.
\end{proposition}

\begin{proof}
    Consider using the adjoint representation. The kernel of $\ad$
    \[
        \ker(\ad) = \{ x \in L \mid \ad(x) = 0 \} = \{ x \in L \mid [x, z] = 0, \forall z \in L \} = Z(L)
    \]
    by Remark \ref{rmk: extending morphism results} is an ideal in $L$. Since $L$ is simple, $\ker(\ad) = 0$. Using \hyperref[rmk: extending morphism results]{First Isomorphism Theorem}, $L/\ker(\ad) \simeq L \simeq \im(\ad)$. Since $\ad: L \to \gl(V)$, $\im(\ad) \subseteq \gl(L)$. By Definition \ref{def: linear lie algebra} $L$ is a linear Lie algebra.
\end{proof}

\begin{notation}
    It is often denoted $\GL(V) = \{ g \in \gl(V) \mid g \text{ invertible} \}$. This coincides with the general linear group.
\end{notation}
\nogap
\begin{example}[Adjoint Representation of $\GL(V)$]
    Let $g \in \GL(V)$. Define $\Ad(g): \gl(V) \to \gl(V)$ by $\Ad(g)(x) = gxg^{-1}$, which is the adjoint representation of $\GL(V)$.
\end{example}

\begin{definition}[Nilpotent (Element)]
    Assume that $\fchar{\F} = 0$. Let $L$ be a Lie algebra on $\F$. Then $x \in L$ is \textbf{nilpotent} if $(\ad(x))^k = 0$ for some $k \in \Z_{\geq 0}$.
\end{definition}

\begin{definition}[Exponential Map]
    Let $L$ be a Lie algebra, and $x \in L$ nilpotent of order $k$. Then the \textbf{exponential map} is defined by
    \[
        \exp(\ad(x)) = \sum_{n = 0}^{k-1} \frac{(\ad(x))^n}{n!}
    \] 
    which coincides with the Taylor expansion. All terms of order $k$ vanishes due to $x$ being $k$-nilpotent.
\end{definition}

\begin{lemma}
    $\exp(\ad(x)) \in \Aut(L)$. More generally, if $\delta \in \Der(L)$ is nilpotent, then $exp(\delta) \in \Aut(L)$.
\end{lemma}

\begin{proof}
    We seek to prove only the general version of the statement. First verify that $\ad$ is a derivation, i.e. we have the equality $\ad([x, y])(z) = ([x, \ad(y)] + [\ad(x), y])(z)$ for all $x, y, z \in L$. This is a direct consequence of the Jacobi Identity:
    \[
        \ad([x, y])(z) = [[x, y], z] = [x, [y, z]] - [y, [z, x]] = [x, \ad(y)(z)] - [y, \ad(x)(z)] = ([x, \ad(y)] + [\ad(x), y])(z)
    \]
    Now for the general statement on derivations, first verify that $\delta$ is a homomorphism. Write out the expressions explicitly:
    \begin{align*}
        \exp(\delta(x)) \exp(\delta(y))
        & = \left( \sum_{n = 0}^{k-1} \frac{\delta^n(x)}{n!} \right) \left( \sum_{m = 0}^{k-1} \frac{\delta^m(y)}{m!} \right) \\
        & = \sum_{n = 0}^{2k-2} \left( \sum_{i = 0}^n \frac{\delta^i(x)}{i!} \cdot \frac{\delta^{n-i}(y)}{(n-i)!} \right) \\
        & = \sum_{n = 0}^{2k-2} \frac{\delta^n(xy)}{n!} & \text{(Leibniz Rule)} \\
        & = \sum_{n = 0}^{k-1} \frac{\delta^n(xy)}{n!} = \exp(\delta(xy)) & \text{($\delta^k = 0$)}
    \end{align*}
    The fact that $\exp(\delta)$ is a bijection can be verified via writing out its inverse: let $\exp(\eta) = \Id - \eta$. Then $\exp^{-1} = \sum_{n=0}^{k-1} \eta^n$, where we designate $\eta^0 = \Id$. 
\end{proof}

\begin{example}
    Let $\F$ be a field of characteristic $0$, and $L = \F^2$ with basis $\{v_1, v_2\}$. Endow it with a Lie algebra structure by setting $[v_1, v_2] = v_1$. Then $\ad \in \gl_2(L)$, with the matrix representation
    \[
        \ad(v_1) = 
        \left(
        \begin{array}{c|c}
            \ad(v_1)(v_1) & \ad(v_1)(v_2) \\
        \end{array}
        \right)
        = 
        \begin{pmatrix}
            0 & 1 \\
            0 & 0
        \end{pmatrix}
        \quad
        \ad(v_1)^2 = 0
    \]
    Since $\ad$ is 2-nilpotent, 
    \[
        \exp(\ad(v_1)) = \Id_2 + 
        \begin{pmatrix}
            0 & 1 \\
            0 & 0
        \end{pmatrix}
        =
        \begin{pmatrix}
            1 & 1 \\
            0 & 1
        \end{pmatrix}
    \]
    Using the same notation as in the proof above,
    \[
        \eta =
        \begin{pmatrix}
            0 & -1 \\
            0 & 0
        \end{pmatrix}
        \quad \implies \quad
        \exp^{-1}(\ad(v_1)) =
        \begin{pmatrix}
            1 & -1 \\
            0 & 1
        \end{pmatrix}
        =
        \begin{pmatrix}
            1 & 1 \\
            0 & 1
        \end{pmatrix}^{-1}
    \]
    \ 
\end{example}

\section{Criterion for Nilpotent Lie Algebra}

\begin{definition}[Derived Series]\label{def: derived series}
    Given a Lie algebra $L$, the \textbf{derived series} of $L$ is the sequence of ideals
    \[
        L^{(0)} = L, \quad L^{(1)} = [L, L], \quad \cdots \quad L^{(n)} = [L^{(n-1)}, L^{(n-1)}]
    \]
\end{definition}
\nogap
\begin{remark}
    Similar to the cases in group or ring theory, the relation of ideals is in general not transitive. That is, directly from the expression we know that for all $n$, $L^{(i)}$ is an ideal in $L^{(i-1)}$; but in general $L^{(n)}$ is not an ideal in $L$ for $n \in \Z_{\geq 2}$.
\end{remark}
\nogap
\begin{definition}[Solvable]
    A Lie algebra $L$ is \textbf{solvable} if $L^{(n)} = 0$ for some $n$.
\end{definition}

\begin{example}\label{ex: derived series}
    Fix $\F$ to be a field. Let $\liealg{t}_n(\F)$ be upper triangular $n$-by-$n$ matrices over $\F$, $\liealg{d}_n(\F)$ be diagonal $n$-by-$n$ matrices and $\liealg{n}_n(\F)$ strictly upper triangular $n$-by-$n$ matrices. Then $\liealg{d}_n(\F)$ is solvable since in general diagonal matrices commutes with all matrices in terms of multiplication (i.e. lies in the center of the Lie algebra, giving $\liealg{d}_n(\F)^{(1)} = [\liealg{d}_n(\F), \liealg{d}_n(\F)] = 0$). $\liealg{n}_n(\F)$ is also solvable since strictly upper triangular matrices are nilpotent. Furthermore, as vector spaces $\liealg{t}_n(\F) = \liealg{d}_n(\F) \oplus \liealg{n}_n(\F)$, giving $\liealg{t}_n(\F)$ solvable.

    In general we can express the $k$-th derived algebra of $\liealg{t}_n(\F)$ as
    \[
        \liealg{t}_n(\F)^{(k)} = \{ (g_{ij}) \mid g_{ij} = 0,\ \forall i + k - 1 \geq j \}
    \]
    which is nilpotent of order $n-1$.
\end{example}
\nogap
\begin{example}
    All simple Lie algebras are non-solvable. By definition $[L, L]$ is an ideal in $L$; and $L$ being simple implies that $[L, L]$ must be either 0 or $L$. However being simple also requires that $L$ is not abelian. Therefore $L^{(1)} = [L, L] = L$, giving $L^{(k)} = L$ for all $k$ and therefore is non-solvable. One of such examples is $\liealg{sl}_n(\F)$ (also mentioned but not proved in \ref{ex: simple lie algebra}).
\end{example}

\begin{remark}\label{rmk: extending solvability}
    Similar to solvability in group theory, we have the following results:
    \begin{enumerate}
        \item If $L$ is solvable, then so are its subalgebras and $\im \phi$ for all $\phi: L \to L'$ homomorphisms.
        \item For $I \subseteq L$ an ideal, if both $I$ and $L/I$ are solvable, then $L$ is solvable.
        \item If $I$ and $J$ are solvable ideals in $L$, then $I + J$ is also solvable.   
    \end{enumerate}
\end{remark}

\begin{corollary}
    If $L$ is solvable, then $L$ has a unique maximal solvable ideal, called the \textbf{radical ideal}, denoted $\Rad L$.
\end{corollary}

\begin{proof}
    Let $S$ be a maximal solvable ideal of $L$, and $I$ any solvable ideal of $L$. There fore $S \subseteq I + S$, and by Remark \ref{rmk: extending solvability} $I + S$ is solvable. Maximality of $S$ implies $I + S = S$, and therefore $I \subseteq S$ which verifies the uniqueness. Existence follows from the fact that the zero ideal is solvable.
\end{proof}

\begin{definition}[Semisimple (Algebra)]
    A non-zero Lie algebra $L$ is \textbf{semisimple} if $\Rad(L) = 0$, i.e. it does not admit a nontrivial solvable ideal.
\end{definition}

\begin{remark}
    We have the following properties directly resulting from the definition:
    \begin{enumerate}
        \item A simple Lie algebra is semisimple (as expected), as the radical ideal is an ideal.
        \item If a Lie algebra $L$ is not solvable, then $L/\Rad L$ is semisimple, as by Correspondence Theorem a solvable ideal $J \subseteq L/\Rad L$ corresponds to a solvable ideal $I \subseteq L$ containing $\Rad L$. From definition of radical ideals $I = J$, and therefore $L/\Rad L$ is semisimple.
    \end{enumerate}
\end{remark}

\begin{definition}[Descending (Lower) Central Series]
    Given a Lie algebra $L$, its \textbf{descending central series} (or \textbf{lower central series}) is the sequence of ideals
    \[
        L^{0} = L, \quad L^{1} = [L, L], \quad \cdots \quad L^{n} = [L, L^{n-1}]
    \]
    The notation is consistent if we view Lie brackets in Lie algebras as multiplication.
\end{definition}
\nogap
\begin{remark}
    Different from the \hyperref[def: derived series]{derived series}, every term in the descending central series is an ideal in $L$.
\end{remark}

\textstart
With the notion of ``raising a power'' for Lie algebras, we can extend the definition of nilpotency:

\begin{definition}[Nilpotent (Lie Algebra)]
    A Lie algebra $L$ is \textbf{nilpotent} if $L^{n} = 0$ for some $n$.
\end{definition}
\nogap
\begin{example}
    Revisiting the example for derived series (Example \ref{ex: derived series}) with the same notation, although similarly both $\liealg{n}_n(\F)$ and $\liealg{d}_n(\F)$ are nilpotent, $\liealg{t}_n(\F)$ is not nilpotent, as it contains specifically diagonal matrices. The explicit expression for $\liealg{n}_n(\F)$ agrees the nilpotency of upper triangular matrices:
    \[
        \liealg{n}_n(\F)^k = \{ (g_{ij}) \mid g_{ij} = 0,\ \forall i + k \geq j \}
    \]
\end{example}

\begin{remark}
    Similarly, the properties of nilpotent groups extend to nilpotent Lie algebras:
    \begin{enumerate}
        \item If a Lie algebra $L$ is nilpotent, and all its subalgebras and $\im \phi$ for all $\phi: L \to L'$ homomorphism are nilpotent. 
        \item If $L/Z(L)$ is nilpotent, then $L$ is nilpotent.
        \item If $L \neq 0$ is nilpotent, then $Z(L) \neq 0$.
    \end{enumerate}
\end{remark}

\begin{definition}[Ad-nilpotent]
    Given a Lie algebra $L$, an element $x \in L$ is \textbf{ad-nilpotent} if $\ad(x)$ is nilpotent.
\end{definition}

\begin{lemma}\label{lem: nilpotent algebra implies every element is ad-nilpotent}
    If $L$ is nilpotent, then for all $x \in L$, $x$ is ad-nilpotent.
\end{lemma}

\begin{proof}
    Since $L$ is nilpotent, there exists $n$ s.t. $L^n = 0$. For all $x, y \in L$, since $(\ad(x))^n (y) \in L^n$, $(\ad(x))^n = 0$.
\end{proof}

\textstart
We now seek prove the converse. From now on, assume that $\F$ is a field of characteristic and algebraically closed. Our main focus will be Lie algebras over such fields. For simplicity, when referring to the dimension of a Lie algebra, we will simply use $\dim (-)$ instead of $\dim_{\F} (-)$.

\begin{theorem}[Engel]\label{thm: Engel}
    Let $L$ be a Lie algebra. If for all $x \in L$, $x$ is ad-nilpotent, then $L$ is nilpotent.
\end{theorem}

\begin{lemma}\label{lem: nilpotent implies ad-nilpotent}
    If $x \in \gl(V)$ is a nilpotent endomorphism, then $\ad(x)$ is nilpotent.
\end{lemma}

\begin{proof}
    This can be verified via writing out the expression explicitly. Using induction, we have
    \[
        \ad^n(x) (y) = \sum_{i = 0}^n \binom{n}{i} x^i y x^{n-i}
    \]
    Suppose that $x$ is $k$-nilpotent. Choosing $n = 2k$ makes $\ad^n(x)$ vanish.
\end{proof}

To prove Engel's theorem, we first prove the following theorem:

\begin{theorem}\label{thm: 0 is eigenvalue of lie algebra of nilpotent endomorphisms}
    Let $L$ be a subalgebra of $\gl(V)$, for some vector space $V$ where $\dim V$ is finite. If $L$ consists of nilpotent endomorphisms and $V \neq 0$, then there exists a nonzero $v \in V$ s.t. $L \cdot v = 0$, i.e. 0 is always an eigenvalue of $L$.
\end{theorem}
\nogap
\begin{remark}
    This is the analogy of a similar (yet under that context much easier to prove) result in linear algebra: given a homomorphism that is not surjective, the kernel is non-trivial; and therefore the dimension of the image is strictly less than the dimension of the domain.
\end{remark}

\begin{proof}
    Apply induction on the dimension of $L$:
    \begin{itemize}
        \item $\dim L = 0$. Then any nonzero $v \in V$ works (the result is vacuous).
        \item $\dim L = 1$. Then $L$ can be written as the linear span of a single element $L = \linspan{\ell}$. $\ell$ being nilpotent implies that its only eigenvalue is 0. Therefore for any $v$ being an eigenvector of $L$, $\ell v = 0 \implies L \cdot v = 0$.
        \item \emph{Inductive Step.} Let $K \subsetneq L$ be a maximal subalgebra (This can be 0, which degenerates to the case of $\dim L = 0$). $K \neq L$ implies that $\dim K < \dim L$, giving $\dim (L/K) < \dim L$. Now take $V = L/K$, and apply the inductive hypothesis: there exists a nonzero $v + K \in L/K$ s.t. $(L/K) \cdot (v + K) = 0$, i.e. $L\cdot v \in K$. Recall that the action of Lie algebra on itself is via adjoint representation, i.e. we have $[y, x] \in K$ for all $y \in K$ and $x \in L$. Since $K$ is maximal in $L$, $\dim K = \dim L - 1$. Therefore, we can write $L = \linspan{K, x}$. By the inductive hypothesis, there exists a nonzero $v \in V$ s.t. $K \cdot v = 0$. Denote all such elements by $W := \{ v \in V \mid K\cdot v = 0 \}$. We now seek to prove that this set is also annihilated by $L$. 
        
        First observe that for all $k \in K, w \in W$, $kxw = x(kw) - [x, k]w$. Inductive hypothesis on $K$ implies that $kw = 0$ and therefore $x(kw) = 0$; and inductive hypothesis on $L/K$ implies that $[x, k]w = 0$ for all $x \in L$. By definition of $W$, this gives $x \cdot w \in W$, i.e. $\linspan{x}$ is a subalgebra of $\gl(W)$. Since $\linspan{x}$ is 1-dimensional, IH implies that there exists $w_0 \in W$ s.t. $x \cdot w_0 = 0$. Then decomposing application of $L$ on $w_0$ gives $L \cdot w_0 = K \cdot w_0 + \F x \cdot w_0 = 0$.
    \end{itemize}
\end{proof}

\textstart
Using this theorem it is then straightforward to prove Engel's theorem:

\begin{proof}[Proof of Theorem \ref{thm: Engel}]
    Similarly apply induction on the dimension of $L$:
    \begin{itemize}
        \item $\dim L = 1$. In this case, $L$ is abelian, i.e. $L^2 = [L, L] = 0$. By definition $L$ is nilpotent.
        \item \emph{Inductive step.} Suppose that the result holds for $V$ s.t. $\dim V \leq n$. By hypothesis, for all $x \in L$, $\ad(x)$ is nilpotent. Use the notation of Theorem \ref{thm: 0 is eigenvalue of lie algebra of nilpotent endomorphisms}, letting $V = L$ gives that $\ad(L)$ is a subalgebra of $\gl(L) = \gl(V)$. The result of Theorem \ref{thm: 0 is eigenvalue of lie algebra of nilpotent endomorphisms} gives that there exists $x \in L$ s.t. $\ad(L)(x) = [L, x] = 0$. This implies that $Z(L) \ni x$ is nonzero. Since $Z(L)$ is a vector subspace, $\dim (L/Z(L)) < \dim L$. Inductive hypothesis gives that under the hypothesis, $\dim (L/Z(L))$ is nilpotent. Furthermore, $Z(L)$ is nilpotent by definition (as $(Z(L))^2 = 0$). By Remark \ref{rmk: extending solvability}, $L$ is nilpotent.
    \end{itemize}
\end{proof}

\begin{definition}[Flag]
    Assume that $\dim V$ is finite. A \textbf{flag} in $V$ is a sequence of subspaces
    \[
        0 = V_0 \subseteq V_1 \subseteq \cdots \subseteq V_n = V \quad \text{s.t.} \quad \dim V_i = i
    \]
\end{definition}
\nogap
\begin{definition}[Stabilizer on Flags]
    An element $x \in \End(V)$ \textbf{stabilizes} a flag if $x(V_i) \subseteq V_i$ for all $i$.
\end{definition}

\textstart
Using ``flags'' we can describe the theorem above in a language more resembling that in vector spaces:

\begin{corollary}
    If $L \in \gl(V)$ consists of nilpotent endomorphisms, then there exists a flag $(V_i)$ stable under $L$, i.e. in such basis $L \in \liealg{n}_n(\F)$, which can be expressed by a strictly upper triangular matrix.
\end{corollary}

\begin{proof}
    It suffices to give a construction of such flag. Theorem \ref{thm: 0 is eigenvalue of lie algebra of nilpotent endomorphisms} implies that there exists nonzero $v \in V$ s.t. $L\cdot v = 0$. Set $V_1 = \F v$ and $W = V / V_1$. Repeat the process until $W = 0$; and ``flatten'' the quotient spaces by taking the preimage of the quotient map, and consider the elements as vector subspaces of $V$. 
\end{proof}

\begin{lemma}
    Let $L \neq 0$ be a nilpotent Lie algebra, and $I \subseteq L$ a nonzero ideal. Then $I \cap Z(L) \neq 0$.
\end{lemma}

\begin{proof}
    $L$ being nilpotent implies that the derived algebra $L' = [L, L] \subsetneq L$, i.e. $Z(L) \neq 0$. Consider again the action of $L$ on itself via adjoint. Apply again Theorem \ref{thm: 0 is eigenvalue of lie algebra of nilpotent endomorphisms} with $V = I$, there exists $x \in I$ s.t. $\ad(L)(x) = 0$, i.e. $x \in Z(L)$. This gives a nonzero element in $I \cap Z(L)$.
\end{proof}

\begin{theorem}\label{thm: solvable implies common eigenvector}
    Let $L$ be a solvable Lie subalgebra of $\gl(V)$, and assume that $\dim V$ is finite. If $V \neq 0$, then $V$ contains a common eigenvector (but the eigenvalue may change), i.e. there exists $v \in V$ s.t. for all $\ell \in L$, $\ell v = \lambda(\ell) v$ where $v$ is an eigenvector of $\ell$.
\end{theorem}

\begin{proof}
    Proceed via applying induction on $\dim L$. The case of $\dim L = 1$ is given by any scalar. 
    
    First sketch the framework of the proof:
    \begin{itemize}
        \item First define an ideal $I$ of codimension 1 in $L$. Use the inductive hypothesis to get a common eigenvector $v_0$.
        \item Verify that $L$ stabilizes a certain space $W$ containing $v_0$. 
        \item Find an element $z \in W$ s.t. $L = I + \F z$.
    \end{itemize}
    Now proceed the proof:
    \begin{itemize}
        \item \bu{Step 1}. Since $\dim L > 0$ and $L$ is solvable, $[L, L] \subsetneq L$ (otherwise the derived series will have identical terms and thus do not terminate, contradicting the hypothesis that $L$ is solvable). By definition $L/[L, L]$ is abelian.\footnote{This quotient is to ensure that there exists such a codimension-1 subspace.} Then any subspace $I \subseteq L/[L, L]$ will be an ideal. Take $I' \subseteq L/[L, L]$ be an ideal of codimension 1. Take its preimage $I \subseteq L$ of the quotient map, which is an ideal of codimension 1 in $L$. Inductive hypothesis implies that there exists nonzero $v \in V$ s.t. for all $x \in I$, $x \cdot v = \lambda(x)v$. Fix $\lambda$ and define $W := \{ w \in V \mid x \cdot w = \lambda w, \forall x \in I \}$. $v \in V$ implies that $W$ is nontrivial and fixed by $I$.
        \item \bu{Step 2}. We now want to show that the space is actually fixed by $L \supset I$. Use the same trick as the one in the previous proof: Let $w \in W, x \in L$; by definition, for all $y \in I$ we have the equality
        \begin{equation}\tag{$\ast$}\label{eq: transforming L, I application}
            (yx) w = xyw - [x, y]w = \lambda(y) xw - \lambda([x, y]) w
        \end{equation}
        To show that $xw \in W$, it suffices to show that $y xw = \lambda(y) xw$, i.e. $\lambda([x, y]) = 0$. Let $n > 0$ be the smallest integer s.t. $\{ w, xw, \cdots, x^{n-1}w \}$ are linearly independent. Define $W_i = \linspan{w, xw, \cdots, x^{i-1}w}$. It is clear that there are finitely many distinct $W_i$s, as $W_{n+j} = W_n$ for all $j \in \Z_{\geq 0}$; and $x W_n \subseteq W_n$. Eq. \eqref{eq: transforming L, I application} implies that $y x^i w = \lambda(y) x^i w$ for all $i$ (as $x^i = 0$), i.e. $I W_n \subseteq W_n$. That is, $y$ is an upper triangular matrix w.r.t. the basis $\{ w, xw \cdots, x^{n-1}w \}$; and diagonal entries are all $\lambda(y)$, i.e. for all $y \in I, \Tr_{W_n}(y) = n \lambda(y)$. Since $I$ is an ideal, $y \in I \implies [x, y] \in I$, i.e. $\Tr_{W_n}([x, y]) = \lambda([x, y])$. However, $\Tr_{W_n}([x, y]) = 0$ as $\Tr_{W_n}(xy) = \Tr_{W_n}(yx)$ for all $x, y \in L$, giving $\lambda([x, y]) = 0$.
        \item \bu{Step 3}. Since $I$ is codimension-1, we can write $L = I + \F z$ for some $z \in L$. Since $\F$ is assumed to be algebraically closed, there exists eigenvector $v_0 \in W$ s.t. $z v_0 = c v_0$ for all $z \in L$ with the corresponding $c \in \F$. Furthermore by definition of $W$, for all $y \in I$, $y \cdot v_0 = \lambda(y) v_0$, i.e. $v_0$ is a common eigenvector of $L$. 
    \end{itemize}
\end{proof}

With the above result, we can state the following theorem (as a corollary):

\begin{theorem}[Lie]\label{thm: Lie}
    Let $L$ be a solvable Lie subalgebra of $\gl(V)$ with $\dim V$ finite. Then $L$ stabilizes a flag of $V$, i.e. there exists a basis of $V$ for which $L$ can be represented as an upper triangular matrix.
\end{theorem}

\begin{proof}
    Construct the basis by repeatedly including the common eigenvectors of $L$.
\end{proof}

\begin{corollary}
    Let $L$ be an $n$-dimensional Lie algebra. Then there exists ideals $0 = I_0 \subseteq I_1 \subseteq \cdots \subseteq I_n = L$ s.t. $\dim I_i = i$.
\end{corollary}

\begin{proof}
    Consider $\ad: L \to \gl(L)$. By Theorem \ref{thm: Lie}, $\ad(L)$ stabilizes a flag of $L$. Definition of $\ad$ ensures that they are ideals.
\end{proof}

\section{Criterion for Solvable Lie Algebra}

\begin{definition}[Semisimple (Endomorphism)]
    Let $V$ be a finite dimensional vector space over $\F$. $x \in \End(V)$ is \textbf{semisimple} if the roots of its minimal polynomial over $\F$ are distinct. Assuming $\F$ is algebraically closed, this is equivalent to $x$ being diagonalizable.
\end{definition}

\begin{proposition}[Jordan Decomposition]\label{prop: Jordan Decomposition}
    For every $x \in \End(V)$, there exists unique $x_s, x_n \in \End(V)$ s.t. $x = x_s + x_n$ where $x_s$ is semisimple, $x_n$ is nilpotent, and $x_s x_n = x_n x_s$. Furthermore, we have the following properties:
    \begin{enumerate}[label=\arabic*)]
        \item There exists $p, q \in \F[T]$ s.t. $p(0) = q(0) = 0$; and $x_s = p(x)$, $x_n = q(x)$ for that specific $x \in \End(V)$ above.
        \item If we have $A \subseteq B \subseteq V$ subspaces, and $xB \subseteq A$, then both $x_s B \subseteq A$, and $x_n B\subseteq A$.
    \end{enumerate}
\end{proposition}

\begin{proof}
    Let $a_1, \dots, a_k$ be distinct eigenvalues of $x$, with multiplicities $m_1, \dots, m_k$. By definition the characteristic polynomial is given by $\operatorname{char}(x) = \prod_{i = 1}^k (T - a_i)^{m_i}$. This allows decomposing $V$ into eigenspaces: denoting $V_i = \ker (T - a_i)^{m_i}$, we have $V = \bigoplus_{i = 1}^k V_i$; and $x$ preserves $V_i$. Furthermore, the characteristic polynomial of the endomorphism of $x$ on $V_i$ is given by the corresponding part of the characteristic polynomial $\fchar_{V_i}(x) = (T - a_i)^{m_i}$.

    By Chinese Remainder Theorem gives the isomorphism
    \[
        \F[T]/\operatorname{char}(x) \simeq \prod_{i = 1}^k \F[T]/\operatorname{char}_{V_i}(x)
    \]
    This is applicable as the eigenvalues $a_i$ are distinct, indicating that the ideals generated by $\operatorname{char}_{V_i}(x)$ respectively are coprime. Furthermore there exists $p \in \F[T]$ s.t. for all $i$, $p \equiv a_i \mod{\operatorname{char}_{V_i}(x)}$; and $p \equiv 0 \mod{T}$. The second condition can be achieved via noticing that the ideals generated by $T$ and $(T - a_i)^{m_i}$ are coprime if $a_i \neq 0$; otherwise the second condition is implied by the first one. 

    Set $q := \Id_{\F[x]} - p$ (where we view $x \in \F[x]$), and let $x_s = p(x)$, $x_n = q(x)$. Since we require $p \equiv 0 \mod{T}$, $p(0) = 0$. Then $q(0) = \Id_{\F[x]}(0) - p(0) = 0$. This also implies 2). Since $p$ and $q$ commute (as elements of $\F[T]$ commute), $x_s$ and $x_n$ commute. 
    
    By definition of eigenspaces, $V_i$ is stabilized by $x$ for all $i$. This in particular implies that both $p(x)$ and $q(x)$ stabilize $V_i$ for all $i$. As previously we have shown that $p \equiv a_i \mod{\operatorname{char}_{V_i}(x)}$, $\restr{p(x)}{V_i} = a_i \Id_{V_i}$, i.e. the matrix representation of $p$ consists of only diagonal entries, which implies that $p$ is semisimple. By definition, $x_n = x - x_s$, giving $x_n^m = (x - x_s)^m \equiv (x - a_i)^m \mod{V_i}$, giving that $x_n$ is $m_i$-nilpotent when restricted to $V_i$. Setting $\widetilde{m} = \max_i \{ m_i \}$, we have $x_n$ is $\widetilde{m}$-nilpotent. 

    It then remains to show that the decomposition $x = x_s + x_n$ is unique. Suppose that we have another decomposition $x = x_s' + x_n'$, then $x_s - x_s' = x_n' - x_n$, which is both semisimple and nilpotent. This can only be represented by the zero matrix, i.e. $x_s = x_s'$, and $x_n = x_n'$.
\end{proof}

\begin{lemma}\label{lem: semisimple implies adjoint is semisimple}
    Similar to Lemma \ref{lem: nilpotent implies ad-nilpotent}, if $x \in \gl(V)$ is semisimple for some vector space $V$, then so is $\ad(x)$.
\end{lemma}

\begin{proof}
    Use the fact that semisimple endomorphisms are diagonalizable. Fix a basis $\{ v_1, \dots, v_n \}$ of $V$ s.t. the matrix representation of $x$ is $\diag( a_1, \dots, a_n )$. Denote the basis of $\gl(V)$ by $\{ e_{ij} \}$. Notice $\ad(x)(e_{ij}) = (a_i - a_j) e_{ij}$, which implies that $\ad(x)$ is diagonalizable.
\end{proof}

\begin{proposition}
    Let $x \in \End(V)$ with its Jordan Decomposition $x = x_s + x_n$. Then the Jordan Decomposition of $\ad(x)$ is given by $\ad(x) = \ad(x_s) + \ad(x_n)$.
\end{proposition}

\begin{proof}
    By Lemma \ref{lem: nilpotent implies ad-nilpotent} amd \ref{lem: semisimple implies adjoint is semisimple}, we know $\ad(x_s)$ is semisimple, and $\ad(x_n)$ is nilpotent. From Proposition \ref{prop: Jordan Decomposition} we know that such $\ad(x_s)$ and $\ad(x_n)$ are unique as long as we can show that they commute. This is indeed the case as $\ad(\cdot)$ gives a representation, i.e. the commutator is given by the Lie bracket
    \[
        [\ad(x_s), \ad(x_n)] = \ad([x_s, x_n]) = \ad(0) = 0    
    \]
    as $x_s$ and $x_n$ commute.
\end{proof}

\begin{lemma}\label{lem: prep for Cartan's 1st Criterion}
    Let $A \subseteq B \subseteq \gl(V)$ be subspaces, and define $M := \{ x \in \gl(V) \mid [x, B] \subseteq A \}$. For fixed $x \in M$, if for any other $y \in M$ we have $\Tr(xy) = 0$, then $x$ is nilpotent.
\end{lemma}

\begin{proof}
    Try to use Jordan Decomposition, by showing that the semisimple component is zero. Write $x = x_s + x_n$ as Jordan Decomposition. Fix a basis $\{ v_1, \dots, v_n \}$ of $V$ s.t. $x_s$ can be represented as a diagonal matrix $\diag( a_1, \dots, a_n )$. Since we work under the setting of $\fchar{\F} = 0$, $\Q \hookrightarrow \F$. Let $E \subseteq \F$ be the subspace (viewing $\F$ as a $\Q$-vector space) spanned by the eigenvalues $a_1, \dots, a_n$ over $\Q$. We need to show $E = 0$. Since $V$ is finite dimensional, it suffices to show that its dual $E^* := \Hom(E, \Q) = 0$ as we have the isomorphism $E \simeq E^*$.

    Let $f \in E^*$, and $y \in \gl(V)$ s.t. $y = \diag(f(a_1), \dots, f(a_n))$. Let $\{ e_{ij} \}$ be the standard basis of $\gl(V)$. In the proof of Lemma \ref{lem: semisimple implies adjoint is semisimple}, we have shown that $\ad(x_s)(e_{ij}) = (a_i - a_j) e_{ij}$, giving $\ad(y)(e_{ij}) = (f(a_i) - f(a_j))e_{ij}$. By Lagrange interpolation, there exists $r \in \F[T]$ s.t. $r(a_i - a_j) = f(a_i - a_j)$. By Lemma \ref{lem: semisimple implies adjoint is semisimple}, $\ad(x_s)$ gives the semisimple component of $\ad(x)$, i.e. there exists $p \in \F[T]$ s.t. $p(\ad(x)) = \ad(x_s)$, and $p(0) = 0$ (by Proposition \ref{prop: Jordan Decomposition}). Applying $r$ on both side gives
    \[
        r(p(0)) = 0, \quad r(p(\ad(x)))(e_{ij}) = r(\ad(x_s))(e_{ij}) = (f(a_i) - f(a_j))(e_{ij}) = \ad(y)e_{ij} \implies r(p(\ad(x))) = \ad(y)
    \]
    Now use the hypothesis: suppose that $x \in M$, then we have $[x, B] \subseteq A \implies \ad(x)(B) \subseteq A$. Furthermore, since $r(p(0)) = 0$, i.e. $r \circ p$ has no constant term, $\ad(y) = (r \circ p)(\ad(x))$ applies at least once $\ad(x)$, giving $\ad(y)(B) = (u(\ad(x)))\ad(x)(B) \subseteq (u(\ad(x)))(A) \subseteq A$ for some $u \in \F[T]$, as $A \subseteq B$ implies that $\ad^n(x)(B) \subseteq \ad^{n-1}(x)(A) \subseteq \ad^{n-1}(x)(B)$. This then gives $y \in M$, which by hypothesis implies that $\Tr(xy) = 0$.

    Recall we have chosen the basis s.t. in matrix representation, $x = \diag(a_1, \dots, a_n)$ and $y = \diag(f(a_1), \dots, f(a_n))$. The condition $\Tr(xy) = 0$ then translates to 
    \[
        0 = \sum_{i = 1}^n a_i \cdot f(a_i) = f\left( \sum_{i = 1}^n a_i \cdot f(a_i) \right) = \sum_{i = 1}^n f(a_i) \cdot f^2(a_i) = \sum_{i = 1}^n (f(a_i))^2 \quad \implies \quad f(a_i) = 0, \forall i
    \]
    This then gives $f = 0$.
\end{proof}

\begin{theorem}[Cartan's First Criterion]\label{thm: Cartan's First Criterion}
    Let $L \subseteq \gl(V)$ be a Lie subalgebra. If for all $x \in [L, L]$, $y \in L$, $\Tr(xy) = 0$, then $L$ is solvable.
\end{theorem}

\begin{proof}
    By definition of solvability, it suffices to prove that $[L, L]$ is nilpotent. \hyperref[thm: Engel]{Engel's Theorem} implies that it is sufficient to prove that every $x \in [L, L]$ is nilpotent (as elements in a Lie algebra acts on each other via Lie bracket).

    Now use the result of Lemma \ref{lem: prep for Cartan's 1st Criterion}, with $B = L$, $A = [L, L]$, with $M = \{ x \in \gl(V) \mid [x, L] \subseteq [L, L] \}$. Clearly $L \subseteq M$. To apply the result for lemma, we need to show that for all $x \in [L, L], z \in M$, $\Tr(xz) = 0$. That is, for all $a, b \in L$, we need to show $\Tr([a, b] z) = 0$. Notice
    \begin{equation}\label{eq: transform Lie bracket for trace}
        \Tr([a, b]z) = \Tr(abz - baz) = \Tr(abz) - \Tr(b(az)) = \Tr(abz) - \Tr((az)b) = \Tr(a[b, z]) = \Tr([b, z]a)
    \end{equation}
    by using the property that for all $A, B$, $\Tr(AB) = \Tr(BA)$. Hypothesis gives $\Tr([z, b]a) = 0$ by definition of $W$. Since $[z, b] = -[b, z]$, from Eq. \eqref{eq: transform Lie bracket for trace} we have $\Tr([a, b]z) = 0$. Lemma \ref{lem: prep for Cartan's 1st Criterion} gives the desired result.
\end{proof}

\begin{corollary}\label{cor: corollary of Cartan's First Criterion to ad}
    Let $L$ be a Lie algebra s.t. $\Tr(\ad(x) \ad(y)) = 0$ for all $x \in [L, L]$ and $y \in L$. Then $L$ is solvable.
\end{corollary}

\begin{proof}
    Let $L' = \ad(L)$. This is solvable by applying \hyperref[thm: Cartan's First Criterion]{Cartan's First Criterion}. Further $\ker (\ad(L)) = Z(L)$ which is also solvable as $[Z(L), Z(L)] = 0$. Remark \ref{rmk: extending solvability} implies that $L$ is solvable.
\end{proof}

\section{Criterion for Semisimple Lie Algebra}

\begin{definition}[Killing Form]\label{def: Killing Form}
    The \textbf{Killing form} is a symmetric bilinear form on $L$, defined by
    \[
        \kappa: L \times L \to \F, \quad \kappa(x, y) = \Tr(\ad(x) \ad(y))
    \]
\end{definition}

\begin{lemma}\label{lem: transition of bracket in Killing form}
    $\kappa([x, y], z) = \kappa(x, [y, z])$.
\end{lemma}

\begin{proof}
    Transition by definition to the equality on trace, using Eq. \eqref{eq: transform Lie bracket for trace} and the fact that $\ad$ gives a Lie algebra representation:
    \[
        \kappa([x, y], z)
        = \Tr(\ad([x, y])\ad(z)) = \Tr([\ad(x), \ad(y)]\ad(z)) \overset{\eqref{eq: transform Lie bracket for trace}}{=} \Tr(\ad(x), [\ad(y), \ad(z)]) = \kappa(x, [y, z])
    \]
\end{proof}

\begin{lemma}\label{lem: restricting Killing form on ideal}
    Let $I \subseteq L$ be an ideal, with $K$ being the Killing form of $L$ and $K_I$ being the Killing form of $I$. Then $K_I = \restr{K}{I \times I}$.
\end{lemma}

\begin{proof}
    First notice a general result: let $V$ be a finite dimensional vector space, with $W \subseteq V$ a subspace. If $\phi \in \End(V)$ satisfies $\phi(V) \subseteq W$, then $\Tr(\phi) = \Tr(\restr{\phi}{W})$. This is straightforward via taking a view from the matrix representation, as the diagonal entries corresponding to basis elements in $V \smallsetminus W$ are all zero.

    Since $I$ is an ideal, for all $x \in I$, $\ad(x)(L) \subseteq I$. This gives
    \[
        \kappa_I(x, y) = \Tr(\ad(x) \ad(y)) = \Tr( \restr{(\ad(x) \ad(y))}{I} ) = \Tr(\restr{\ad(x)}{I} \restr{\ad(y)}{I}) = \restr{\kappa(x, y)}{I \times I}
    \]
\end{proof}

\begin{definition}[Radical, Non-Degeneracy]
    Let $\beta$ be a symmetric bilinear form on $L$. The \textbf{radical} of $\beta$ is defined as
    \[
        \Rad(\beta) = \{ x \in L \mid \beta(x, y) = 0, \forall y \in L \}
    \]
    $\beta$ is \textbf{non-degenerate} if $\Rad(\beta) = 0$.
\end{definition}

\begin{lemma}\label{lem: semisimple implies no nontrivial abelian ideals}
    A Lie algebra $L$ is semisimple if and only if $L$ has no nontrivial abelian ideals.
\end{lemma}

\begin{proof}
    Recall that by definition, a Lie algebra $L$ is semisimple if and only if $L$ is nonzero, and the radical of $L$, being the maximal solvable ideal, is 0.
    \begin{itemize}
        \item[$\Rightarrow$] Prove the contrapositive. Suppose that $I \subseteq L$ is a nontrivial abelian ideal. Since all abelian ideals are solvable $([I, I] = I^{(1)} = 0)$, $I \subseteq \Rad(L) \implies \Rad(L) \neq 0$. Then $L$ is not semisimple.
        \item[$\Leftarrow$] Prove also the contrapositive. Suppose that $L$ is not semisimple, then it has a nontrivial radical $\Rad(L)$. By definition there exists $n$ s.t. $\Rad(L)^n = 0$, with $\Rad(L)^{n-1} \neq 0$. $\Rad(L)^{n-1}$ is then a nontrivial abelian ideal.
    \end{itemize} 
\end{proof}

\begin{theorem}\label{thm: semisimple implies non-degenerate Killing form}
    A Lie algebra $L$ is semisimple if and only if the Killing form $\kappa$ of it is non-degenerate.
\end{theorem}

\begin{proof}
    Proceed via showing implication in both directions:
    \begin{itemize}
        \item[$\Rightarrow$] Assume that $L$ is semisimple, i.e. $\Rad(L) = 0$. To show that $\kappa$ is non-degenerate, we need to show that $\Rad(\kappa) = 0$. For all $x \in \Rad(\kappa)$ and $y \in L$, by definition we have $\kappa(x, y) = \Tr(\ad(x) \ad(y)) = 0$. Applying \hyperref[thm: Cartan's First Criterion]{Cartan's First Criterion} with $L = \Rad(\kappa)$ gives $\ad(\Rad(\kappa))$ is solvable. The homomorphism $\ad(\cdot)$ has kernel the center, which is solvable. By Remark \ref{rmk: extending solvability}, $\Rad(\kappa)$ is also solvable. By maximality of $\Rad(L)$, $\Rad(\kappa) \subseteq \Rad(L)$. Therefore, $L$ being semisimple implies that $\Rad(L) = 0$, i.e. $\Rad(\kappa) = 0$.
        \item[$\Leftarrow$] Suppose that $\kappa$ is non-degenerate, i.e. $\Rad(\kappa) = 0$. Let $I \subseteq L$ be an abelian ideal. It suffices to prove tha $I = 0$ from Lemma \ref{lem: semisimple implies no nontrivial abelian ideals}. For all $x \in I$, $y \in L$, we have
        \[
            \ad(x) \ad(y): L \tooh{\ad(y)} L \tooh{\ad(x)} I 
            \quad \implies \quad
            (\ad(x) \ad(y))^2: L \too [I, I] = 0
        \]
        giving $(\ad(x) \ad(y))$ is nilpotent. Then $0 = \Tr(\ad(x) \ad(y)) = \kappa(x, y)$. This implies that $I \subseteq \Rad(\kappa) = 0$.
    \end{itemize}
\end{proof} 

\begin{remark}
    For $\{ x_1, \dots, x_n \}$ any basis of $L$, $\kappa_L$ is non-degenerate if and only if the matrix given by $(\kappa(x_i, x_j))$ is invertible. This can be interpreted as for any basis there will not exists a row/column of zeros in $(\kappa(x_i, x_j))$, matching the definition of non-degeneracy.
\end{remark}

\begin{definition}[Direct Sum]
    A Lie algebra $L$ is a \textbf{direct sum} of ideals, denoted $L = \bigoplus_{i = 1}^n I_i$, if $L$ is the direct sum of $(I_i)$ as vector spaces.
\end{definition}
\nogap
\begin{remark}
    By the definition of direct sum on vector spaces, if we have the direct sum decomposition of $L$ as $(I_i)$, for all $i \neq j$, $I_i \cap I_j = 0$, giving $[I_i, I_j] \subseteq I_i \cap I_j = 0$. Therefore, the Lie bracket on $L$ can be fixed via fixing the Lie bracket on all $I_i$s.
\end{remark}

\begin{theorem}\label{thm: uniqueness of simple ideal decomposition on semisimple lie algebra}
    Let $L$ be a semisimple Lie algebra. Then there exists simple ideals $I_1, \dots, I_n$ s.t. $L = \bigoplus_{i = 1}^n I_i$ which are unique up to isomorphism. Furthermore, $\kappa_{I_i} = \restr{\kappa}{I_i \times I_i}$ for all $i$.
\end{theorem}

\begin{proof}
    Let $I \subseteq L$ be an arbitrary ideal. Define $I^{\perp} := \{ x \in L \mid \kappa(x, y) = 0, \forall y \in I \}$, This is also an ideal, as for all $x \in I^{\perp}$, $y \in L$, $\kappa([x, y], z) = \kappa(x, [y, z]) = 0$ for all $z \in L$ by Lemma \ref{lem: transition of bracket in Killing form}; and the second equality is from the definition of $I^{\perp}$. Applying \hyperref[thm: Cartan's First Criterion]{Cartan's First Criterion} with $L$ being $I$ and $I^{\perp}$ separately gives that both $I$ and $I^{\perp}$ are solvable, hence $I \cap I^{\perp}$ is solvable. $L$ being semisimple implies that $I \cap I^{\perp} \subseteq \Rad(L) = 0$, i.e. $L$ can be decomposed into the direct sum as $L = I \oplus I^{\perp}$. Conduct this process iteratively gives the decomposition of $L$ into simple ideals.

    Now we verify the other desired properties. The decomposition is unique up to isomorphism, as for any $I \subseteq L$ ideal, $[I, L]$ is an ideal of $I$. $L$ being semisimple implies that $I$ is not abelian, hence $[I, L] \neq 0$. $I$ being simple implies $I = [I, L] = \bigoplus_{i = 1}^n [I, I_i]$; but since $[I, I_i]$ is both a sub-ideal in $I$ and $I_i$, with both of them being simple, either $[I, I_i] = 0$ or $[I, I_i] = I_i = I$, where the former implies direct sum decomposition, and the latter implies that $I$ is isomorphic to $I_i$.

    The final claim is proven in Lemma \ref{lem: restricting Killing form on ideal}.
\end{proof}

\begin{corollary}
    The converse of the above theorem is also true, i.e. for $L$ Lie algebra with simple ideals $I_1, \dots, I_n$ s.t. $L = \bigoplus_{i = 1}^n I_i$, $L$ is semisimple. The proof is done by reversing the claims in the proof of the theorem.
\end{corollary}

\section{Lie Algebra Representations}

\textstart
We have seen previously how Lie algebras act on other structures, of which the most frequently used one is the adjoint representation. In this section we formalize the notion. In this section we will assume that the underlying field of the Lie algebra $\F$ is characteristic-0 and algebraically closed. First we review the basics of representation theory:

\begin{definition}[Modules on Lie Algebra]
    Given a Lie algebra $L$, an \textbf{$L$-module} is a vector space $V$ over $\F$ equipped with a map $L \times V \to V, (x, v) \mapsto xv$ s.t. it is distributive in $L$ and $V$, with the scalar multiplication compatible with Lie bracket, i.e.
    \[
        [x, y](v) = x(y(v)) - y(x(v)), \forall x, y \in L, v \in V
    \]
    This is simply the extension of modules where the multiplication on the ring is replaced by the Lie bracket. The morphism between $V, W$ as $L$-modules is a linear map $f: V \to W$ s.t. $f(xv) = xf(v)$ for all $x \in L, v \in V$.
\end{definition}
\nogap
\begin{remark}
    $L$-modules are equivalent to $L$-representations (as is in the generalized result from group theory), where $\phi: L \to \gl(V)$ with $xv = \phi(x) \cdot v$ specifies how the Lie algebra acts on the vector space.
\end{remark}

\begin{definition}[Irreducible]
    An $L$-representation $V$ is \textbf{irreducible} if $V \neq 0$, and the only $L$-submodules of $V$ are 0 and $V$.
\end{definition}
\nogap
\begin{definition}[Completely Reducible]\label{def: completely reducible}
    An $L$-representation $V$ is \textbf{completely reducible} if $V$ is a direct sum of irreducible $L$-submodules.
\end{definition}
\nogap
\begin{remark}
    An $L$-representation $V$ is completely reducible if and only if for all $W \subseteq V$ $L$-submodule, there exists another $L$-submodule $W' \subseteq V$ s.t. $V = W \oplus W'$. This can be verified via choosing a basis of $W$ as a vector space, extending that to a basis of $V$, and construct $W'$ accordingly. 
\end{remark}

\begin{example}
    Let $L = \gl_n(\F)$ with $V = \F^n$. The standard representation is given by $\phi(x) = x$ (as endomorphisms), with multiplication given by matrix multiplication. This is an irreducible representation, as for all nonzero $W \subseteq V$, fixing a basis of it $\{w_1, \dots, w_m\}$ and completing it to be a basis of $V$ $\{ w_1, \dots, w_m, v_{m+1}, \dots, v_n \}$, there exists $x \in \gl_n(\F)$ s.t. $x(w_1) = v_n$ which does not fix $W$.
\end{example}

\begin{definition}[Dual Representation]
    Let $V$ be an $L$-module, The \textbf{dual representation} of $V$, denoted $V^*$, is given by $V^* = \Hom(V, \F)$ with the action of $L$ defined as $(x \cdot f)(v) = -f(xv)$ for all $x \in L, v \in V, f \in V^*$.
\end{definition}

\begin{lemma}
    $V^*$ is an $L$-module.
\end{lemma}

\begin{proof}
    Distributivity is clear. It suffices to check the compatibility with Lie bracket, i.e. we want to show that $(([x, y])(f))(v) = ((xy - yx)(f))(v)$. By definition this is given by
    \begin{align*}
        (([x, y])(f))(v) &= -f([x, y]v) = -f(x(y(v))) + f(y(x(v))) \\
        & = -(yx)(f)(v) + (xy)(f)(v) = ((xy - yx)(f))(v)
    \end{align*}
\end{proof}

\begin{definition}[Faithful Representation]
    An $L$-representation $V$ is \textbf{faithful} if the kernel of the representation $\phi: L \to \gl(V)$ is 0.
\end{definition}

\begin{proposition}
    If $L$ is a semisimple Lie algebra, and $\phi: L \to \gl(V)$ is a faithful representation, then the bilinear map 
    \[
        \beta: L \times L \to \F, \quad \beta(x, y) = \Tr(\phi(x) \phi(y))
    \]
    is non-degenerate.
\end{proposition}

\begin{proof}
    Recall that the radical of a bilinear map $\beta$ is defined as
    \[
        \Rad(\beta) = \{ x \in L \mid \beta(x, y) = 0, \forall y \in L \}
    \]
    Apply \hyperref[thm: Cartan's First Criterion]{Cartan's First Criterion} with $L = \phi(\Rad(\beta))$. $x \in \phi(\Rad(\beta))$ in particular implies that $x \in [\phi(\Rad(\beta)), \phi(\Rad(\beta))]$, with which we hav e$\phi(\Rad(\beta))$ is solvable. Since $\phi$ is faithful, $\Rad(\beta) \simeq \phi(\Rad(\beta))$, i.e. $\Rad(\beta)$ is solvable. Since $L$ is semisimple, $\Rad(\beta) \subseteq \Rad(L) = 0$, giving $\Rad(\beta) = 0$, which is equivalent to saying that $\beta$ is non-degenerate.
\end{proof}

\begin{definition}[Dual Basis]
    Let $V$ be a vector space over $\F$, with $\{ v_1, \dots, v_n \}$ a basis of $V$. Let $\beta: V \times V \to \F$ be a bilinear map. Then the \textbf{dual basis} of $\{ v_1, \dots, v_n \}$ with respect to $\beta$ is given by $\{ v_1^*, \dots, v_n^* \}$ s.t. $\beta(v_i, v_j^*) = \delta_{ij}$.
\end{definition}
\nogap
\begin{remark}
    In $\R^n$ with the canonical basis $\{e_1, \dots, e_n\}$, its dual consists of the covectors, with the bilinear map being tensor contraction.
\end{remark}

\begin{definition}[Casimir Element]
    Let $L$ be a Lie algebra, with $\{x_i\}_{i = 1}^n$ a basis of $L$. Let $\beta$ be an arbitrary bilinear map on $L$, and $\{y_i\}_{i = 1}^n$ the dual of $\{x_i\}_{i = 1}^n$ w.r.t. $\beta$. Given a representation $\phi: L \to \gl(V)$, the \textbf{Casimir element} of $\phi$ w.r.t. $\beta$ is defined as
    \[
        C_{\phi}(\beta) = C_{\phi} = \sum_{i = 1}^n \phi(x_i) \phi(y_i) \in \End(V)
    \]
\end{definition}

\begin{lemma}
    $C_{\phi}$ commutes with $\phi(L)$.
\end{lemma}

\begin{proof}
    Let $x \in L$ be an arbitrary element. Let $a_{ij}, b_{ij} \in \F$ s.t.
    \[
        [x, x_i] = \sum_{j = 1}^n a_{ij} x_j, \quad [x, y_i] = \sum_{j = 1}^n b_{ij} y_j
    \]
    By definition of dual basis, 
    \begin{align*}
        a_{ik}
        & = \sum_{j} a_{ij} \delta_{jk} = \sum_{j} a_{ij} \beta(x_j, y_k) \\
        & = \beta([x, x_i], y_k) = -\beta([x_i, x], y_k) & \text{By \hyperref[eq: transform Lie bracket for trace]{$\Tr([x, y]z) = \Tr(x[y, z])$}} \\
        & = -\beta(x_i, [x, y_k]) \\
        & = -\sum_{j} b_{kj} \beta(x_i, y_j) = -b_{ki}
    \end{align*}
    Notice that in general we have the equality
    \begin{equation}\label{eq: [x, yz] = [x, y]z + y[x, z]}
        [x, yz] = xyz - yzx = xyz - yxz + yxz - yzx = [x, y]z + y[x, z]
    \end{equation}
    To prove the lemma, we only need to show that for all $x \in L$, $[\phi(x), C_{\phi}] = 0$. 
    Compute
    \begin{align*}
        [\phi(x), C_{\phi}] = \sum_{i} [\phi(x), \phi(x_i) \phi(y_i)]
        & = \sum_{i} [\phi(x), \phi(x_i)] \phi(y_i) + \phi(x_i) [\phi(x), \phi(y_i)] & \text{By \eqref{eq: [x, yz] = [x, y]z + y[x, z]}} \\
        & = \sum_{i} \phi([x, x_i]) \phi(y_i) + \phi(x_i) \phi([x, y_i]) \\
        & = \sum_{i, j} a_{ij} \phi(x_j) \phi(y_i) + b_{ij} \phi(x_i) \phi(y_j) = 0
    \end{align*}
    where the last equality holds since $a_{ik} = -b_{ki}$.
\end{proof}

\begin{remark}
    Let $\{y_i\}_{i=1}^n$ be the dual basis of $\{x_i\}_{i=1}^n$ w.r.t. the \hyperref[def: Killing Form]{Killing Form}, then we have
    \[
        \Tr(C_{\phi}) = \sum_{i} \Tr(\phi(x_i) \phi(y_i)) = \sum_{i} \kappa(x_i, y_i) = \dim L
    \]
    If $\phi$ is an irreducible representation, then $C_{\phi}$ is equivalent to a scalar (can be expressed with a matrix with only identical diagonal elements). Therefore, $C_{\phi} = \frac{\dim L}{\dim V} \Id_V$. In particular $C_{\phi}$ does not depend on the choice of basis. 

    Furthermore, by applying Schur Lemma, if $\phi$ is irreducible, then the only endomorphisms of $V$ commuting with $\phi$ are $\F \Id_V$.
\end{remark}

\begin{example}
    Let $L = \liealg{sl}_2(\F)$ with $V = \F^2$. The the representation $\phi: L \to \gl(V)$, $\phi(x) = x$ is irreducible. Write in the standard basis
    \[
        \left\{ 
        x = \begin{pmatrix}
            & 1 \\ &     
        \end{pmatrix}, 
        h = \begin{pmatrix}
            1 & \\ & -1
        \end{pmatrix},
        y = \begin{pmatrix}
            & \\ 1 &
        \end{pmatrix}
        \right\}
         = \{ x_1, x_2, x_3 \}
    \]
    with $\kappa(\ell_1, \ell_2) = \Tr(\phi(\ell_1)\phi(\ell_2)) = \Tr(\ell_1 \ell_2)$. The dual basis is then given by $\{ y, \frac{1}{2}h, x \}$. Verify
    \[
        C_{\phi} = \sum_{i=1}^3 \phi(x_i) \phi(y_i) = xy + \frac{1}{2} h^2 + yx = \frac{3}{2} \Id_2
    \]
    This is consistent with the remark above as $\dim L = 3$, $\dim V = 2$.
\end{example}

\begin{lemma}\label{lem: semisimple repr acts trivially on 1-dim repr}
    Let $\phi: L \to \gl(V)$ be a representation of a semisimple Lie algebra $L$. Then $\phi(L) \subseteq \liealg{sl}(V)$. In particular, $L$ acts trivially on any 1-dimensional $L$-module as $\liealg{sl}(V)$ does.
\end{lemma}

\begin{proof}
    Since $L$ is semisimple, there is no nontrivial solvable ideal, giving $[L, L] = L$. Notice that $[\gl(V), \gl(V)] = \liealg{sl}(V)$ (as $\Tr(AB) = \Tr(BA)$, giving the sum of diagonal elements are 0; and writing out the full matrix gives the equality.) Then
    \[
        \phi(L) = \phi([L, L]) = [\phi(L), \phi(L)] \subseteq \liealg{sl}(V)
    \]
\end{proof}

\begin{theorem}[Weyl]
    Let $L$ be a nonzero semisimple Lie algebra, and $\phi: L \to \gl(V)$ a finite dimensional representation. Then $\phi$ is \hyperref[def: completely reducible]{completely reducible}.
\end{theorem}

\begin{proof}
    It is enough to show that for all $L$-submodule $W \subseteq V$, there exists an $L$-submodule $W' \subseteq V$ s.t. $V = W \oplus W'$. If this holds, then by applying the argument iteratively we can decompose $V$ into a direct sum of irreducible $L$-submodules.

    \bu{Case 1} First assume that $\dim V/W = 1$. 
    \begin{addmargin}{2em}
        \bu{1. a)} Suppose that $W$ is irreducible. Without loss of generality, we can assume that $\phi$ is faithful. (Otherwise, we can consider in turn the Lie algebra $L/\ker \phi$, which is also semisimple by correspondence according to the quotient map.) By Lemma \ref{lem: semisimple repr acts trivially on 1-dim repr}, $L$ acts trivially on $V/W$. Since by definition $L \cdot W \subseteq W$, $C_{\phi} = \sum_{i=1}^n \phi(x_i) \phi(y_i)$ also preserves $W$. Since $W$ is irreducible, $C_{\phi}$ acts as a scalar on $W$. Since $\phi$ is faithful, $C_{\phi} \neq 0$. $\ker (C_{\phi})$ gives an $L$-submodule of $V$. As $L$ acts trivially on $V/W$, so does $C_{\phi}$, and therefore $\ker(C_{\phi})$ is 1-dimensional. $C_{\phi}$ acts as a scalar on $W$ implies that $W \cap \ker(C_{\phi}) = 0$, i.e. we have the decomposition $V = W \oplus \ker(C_{\phi})$.
        
        \bu{1. b)} Suppose that $W$ is not irreducible. Apply induction on dimension of $W$:
        \begin{itemize}
            \item The base case is proven in \bu{1. a)}.
            \item Suppose that the result holds for all $W$ being codimension 1 and irreducible. Let $0 \neq W'' \subseteq W$ be an $L$-submodule. Since $\dim V/W = 1$, after taking the quotient $\dim (V/W'')/(W/W'') = 1$. Furthermore, we have $\dim V/W'' < \dim V$ and $\dim W/W'' < \dim W$. Applying the inductive hypothesis gives that there exists $W'$ where $V/W'' = W/W'' \oplus W'/W''$ where $\dim W'/W'' = 1$. Applying the inductive hypothesis with the larger submodule being $W'$, there exists $X$ where $W' = W'' \oplus X$, and $\dim X = 1$. This gives the decomposition of $V$: $V = W \oplus X$.
        \end{itemize}
    \end{addmargin}
    \bu{Case 2} Now consider the general case, where $W$ is an arbitrary $L$-module. Seek to construct a complement from the First Isomorphism Theorem. Notice that $\Hom(V, W)$ is also an $L$-module, with the scalar multiplication given by $[x, f]v = xf(v) - f(xv)$ for all $x \in L, v \in V, f \in \Hom(V, W)$. Define
    \[
        \mathcal{V} := \{ f \in \Hom(V, W) \mid \restr{f}{W} = a \Id_W, \ a \in \F \}, \quad \mathcal{W} := \{ f \in \Hom(V, W) \mid \restr{f}{W} = 0 \}
    \]
    For all $x \in L$, $f \in \mathcal{V}$, $v \in V$, $[x, f](v) = xf(v) - f(xv) = a(xv) - a(xv) = 0$, implying that $L\mathcal{V} \subseteq \mathcal{W}$. It is clear that $\dim \mathcal{V}/\mathcal{W} = 1$ (parameterized by $a$), giving the decomposition $\mathcal{V} = \mathcal{W} \oplus \mathcal{X}$ where $\dim \mathcal{X} = 1$, where $L$ acts trivially on $X$ (since $L$ is semisimple, and by Lemma \ref{lem: semisimple repr acts trivially on 1-dim repr}). Take arbitrary $g \neq 0$ and $g \in \mathcal{X}$, we can write $V = W \oplus \ker g$, which gives a decomposition.
\end{proof}

\section{Representation of $\liealg{sl}_2(\F)$}

\textstart
Let $L = \liealg{sl}_2(\F)$ where $\F$ is algebraically closed and characteristic-0. Let $V$ be an arbitrary finite-dimensional vector space, and $\phi: L \to \gl(V)$ an $L$-representation. We seek to classify such $V$. 

Recall that the standard basis is given by
\[
    x = \begin{pmatrix}
        0 & 1 \\ 0 & 0     
    \end{pmatrix}, \quad
    h = \begin{pmatrix}
        1 & 0 \\ 0 & -1
    \end{pmatrix}, \quad
    y = \begin{pmatrix}
        0 & 0 \\ 1 & 0
    \end{pmatrix}
\]
which are related via $[h, x] = 2x, [h, y] = -2y, [x, y] = h$.

\section{Root System}

\begin{definition}[Euclidean Space]
    A vector space $E$ is an \textbf{Euclidean space} if it is finite dimensional over $\R$, with an inner product (bilinear, and $(v, v) \geq 0$ for all $v$) $(-, -)$.
\end{definition}

\begin{definition}[Root System]
    A \textbf{root system} $\Phi \subseteq E$ is a finite set in $E$ s.t.
    \begin{enumerate}
        \item 
    \end{enumerate}
\end{definition}

\end{document}
