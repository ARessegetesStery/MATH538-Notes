\documentclass{article}
\usepackage{../header}

\begin{document}

\Makepagesectionhead{MATH 538}{Root System}{ARessegetes Stery}

\tableofcontents  
\clearpage

\section{Root System}

\begin{definition}[Euclidean Space]
    A vector space $E$ is an \textbf{Euclidean space} if it is finite dimensional over $\R$, with an inner product (bilinear, and $(v, v) \geq 0$ for all $v$) $(-, -)$.
\end{definition}
\nogap
\begin{notation}
    Throughout this chapter, we will use parenthesis $(\cdot, \cdot)$ to denote the inner product.
\end{notation}

\begin{definition}[Root System; Roots]
    A \textbf{root system} $\Phi \subseteq E$ is a finite set in $E$ s.t.
    \begin{enumerate}
        \item $0 \notin \Phi$ (in particular $\Phi$ cannot be a subspace of $E$), and $\linspan \Phi = E$.
        \item If $\alpha \in \Phi$, then the only real multiple of $\alpha$ in $\Phi$ is $\pm \alpha$.
        \item For all $\alpha, \beta \in \Phi$, 
        \[
            \inner{\beta, \alpha} := \frac{2(\beta, \alpha)}{(\alpha, \alpha)} \in \Z.
        \]
        \item If $\alpha, \beta \in \Phi$, then $\sigma_{\alpha}(\beta) := \beta - \inner{\beta, \alpha}\alpha \in \Phi$.
    \end{enumerate}
    Elements in a root system are called \textbf{roots}.
\end{definition}

\begin{definition}[Orthogonal Hyperplane]
    The \textbf{orthogonal hyperplane} of $\alpha \in E$ is $P_{\alpha} := \{ \beta \in E \mid (\beta, \alpha) = 0 \}$.
\end{definition}
\nogap
\begin{definition}[Reflection]
    The fourth condition, $\sigma_{\alpha}(\beta)$ is also called the \textbf{reflection}, which essentially reflects $\beta$ w.r.t. the orthogonal hyperplane of $\alpha$.
\end{definition}
\nogap
\begin{remark}
    By the fourth axiom in the definition, for a root system $\Phi$ and any root $\alpha$ in it, $\sigma_{\alpha}(\Phi) = \Phi$.
\end{remark}

\begin{example}[Root System $A_2$]
    Consider the root system $A_2 \subseteq \R^2$ given by $A_2 = \{ \pm \alpha, \pm \beta, \pm (\alpha + \beta) \}$:

    \begin{figure}[H]
        \centering
        \begin{tikzpicture}
            \draw[black, ->] (0, 0) -- (2, 0) node[right] {$\alpha$};
            \draw[black, ->] (0, 0) -- (1, 1.732) node[right] {$\alpha + \beta$};
            \draw[black, ->] (0, 0) -- (-1, 1.732) node[left] {$\beta$};
            \draw[black, dashed, ->] (0, 0) -- (-2, 0);
            \draw[black, dashed, ->] (0, 0) -- (-1, -1.732);
            \draw[black, dashed, ->] (0, 0) -- (1, -1.732);
            \filldraw[black] (0, 0) circle (1pt) node[above] {$\frac{\pi}{3}$};
        \end{tikzpicture}
        \caption{Visualization of the root system $A_2$ in $\R^2$}
    \end{figure}
    It suffices to verify that this root system is closed w.r.t. reflections. Check for example: $\sigma_{\alpha}(\beta) = (\alpha + \beta)$, and $\sigma_{\beta}(\alpha) = -(\alpha + \beta)$. Actually we can read off the reflections from the figure.
\end{example}

\begin{definition}[Weyl Group]
    The \textbf{Weyl Group} $W$ is the subgroup of $\End(E)$ generated by reflections $\{ \sigma_{\alpha} \mid \alpha \in \Phi \}$.
\end{definition}
\nogap
\begin{example}[Weyl Group of $A_2$]
    It is clear that $W(A_2) = \inner{\Id, \sigma_{\alpha}, \sigma_{\beta}, \sigma_{\alpha+\beta}}$. Further notice that
    \[
        \sigma_{\beta} \sigma_{\alpha} (\alpha) = -(\alpha + \beta), \quad 
        \sigma_{\alpha}(\alpha) = -\alpha \quad
        \sigma_{\beta}(\alpha) = \alpha + \beta \quad
        \sigma_{\alpha + \beta}(\alpha) = -\beta
    \]
    which implies that $\sigma_{\beta} \sigma_{\alpha}$ is another distinct element in $W$. Further we could verify that $(\sigma_{\alpha} \sigma_{\beta}) = (\sigma_{\beta} \sigma_{\alpha})^{-1}$. It can be verified that these are the only elements in $W(A_2)$, i.e. we have
    \[
        W(A_2) = \{ 1, \sigma_{\alpha}, \sigma_{\beta}, \sigma_{\alpha + \beta}, \sigma_{\alpha}\sigma_{\beta}, \sigma_{\beta} \sigma_{\alpha} \} \simeq S_3
    \]
\end{example}
\nogap
\begin{remark}
    The example above can be extended to general cases: The root system
    \[
        A_n := \{ \pm (e_i - e_j) \mid 1 \leq i < j \leq (n+1) \}
    \]
    is a root system for 
    \[
        E := \left\{ x = \left( x_i \right)_{i = 1}^{n+1} \in \R^n \left|\ \sum_{i = 1}^{n+1} x_i = 0 \right. \right\}
    \]
    with $W(A_n) \simeq S_{n+1}$.
\end{remark}

\begin{example}[Root System $D_n$]
    The root system $D_n$ is given by $D_n := \{ \pm(e_i \pm e_j) \mid 1 \leq i < j \leq n \}$. The Weyl group for $D_n$ is complicated. The fact is $D_3 \simeq A_3$.
\end{example}
\nogap
\begin{example}[Root System $B_n$ and $C_n$]
    
\end{example}
\nogap
\begin{remark}
    
\end{remark}

\begin{lemma}
    Suppose that $\alpha, \beta \in \Phi$, and $\beta \neq \pm \alpha$. Then $\inner{\alpha, \beta} \inner{\beta, \alpha} \in \{ 0, 1, 2, 3 \}$.
\end{lemma}

\begin{proof}
    By definition, we have
    \[
        \inner{\alpha, \beta} \inner{\beta, \alpha} = \frac{2(\alpha, \beta)}{(\beta, \beta)} \frac{2(\beta, \alpha)}{(\alpha, \alpha)} = \frac{4(\alpha, \beta)(\beta, \alpha)}{(\alpha, \alpha)(\beta, \beta)} \leq 4.
    \]
    by Cauchy-Schwarz inequality. Axiom 3) in the definition gives $\inner{\alpha, \beta} \inner{\beta, \alpha} \in \Z$. Further since $\beta \neq \pm \alpha$ the equality cannot be reached. The result follows.
\end{proof}

\end{document}